\chapter{مقدمه}
\section{مقدمه ای بر \lr{5G} و \lr{6G} }
\lr{6G}
یا 
نسل ششم مخابرات
نشان‌دهنده نسل بعدی فناوری‌های ارتباطی بی‌سیم هستند که انتظار می‌رود با قابلیت‌های پیشرفته‌شان، تجربیات ارتباطی ما را متحول کنند. شبکه‌های
\lr{6G} 
با تکیه بر پایه‌های پیشینیان خود، قصد دارند پیشرفت‌های قابل‌توجهی را از نظر سرعت، ظرفیت، تأخیر و سایر شاخص‌های کلیدی عملکرد ارائه دهند و از این طریق مرزهای ارتباط بی‌سیم را دوباره تعریف کنند. یکی از ویژگی‌های‌ این نسلِ، سرعت بسیار بالای آن خواهد بود. 
این شبکه‌های قدرتمند برای دستیابی به سرعت دانلود و آپلود بی‌سابقه پیش‌بینی می‌شوند و امکان اتصال یکپارچه برای چندین دستگاه را به طور همزمان فراهم می‌کنند. سرعت انتقال داده بسیار زیاد ارائه شده توسط شبکه‌های نسل ششم نه تنها دسترسی سریع‌تر به محتوای دیجیتال را تسهیل می‌کند، بلکه فناوری‌های نوظهوری مانند واقعیت افزوده (\lr{AR})، واقعیت مجازی (\lr{VR}) و پخش ویدئو با کیفیت بالا را تقویت می‌کند و تجربه‌های همه‌جانبه‌ای را برای کاربر فراهم می‌کند.

\lr{5G}،
مخابرات نسل پنجم
سیستمهای بیسیم \LTRfootnote{Wireless} وشبکه‌های مخابراتی بعد از نسل چهارم می‌باشد که تکاملی از لایه‌ی فیزیکی در تکنولوژی شبکه‌های مخابراتی سیار همانند \lr{LTE} است که نسبت به \lr{4G} سرعت و پوشش بهتری را فراهم می‌کند.
\lr{5G}
نوع جدیدی از شبکه را ایجاد می‌کند که به منظور اتصال تقریبا همه و همه چیز با هم از جمله ماشینها، اشیاء و دستگاه‌ها ساخته شده است.
\lr{5G}
 فناوری بی سیم برای ارائه سرعت داده‌های چند گیگابیت بر ثانیه، تأخیر فوق العاده کم، قابلیت اطمینان بیشتر، ظرفیت شبکه گسترده، افزایش در دسترس بودن و تجربه کاربری یکنواخت تر به کاربران بیشتر است. عملکرد بالاتر و بهره وری بهبود یافته باعث افزایش تجربیات کاربر جدید شده و صنایع جدیدی را به هم متصل می‌کند.
 
 
تکنولوژی سیگنال  \lr{5G} برای پوشش فراگیرتر و بازدهی بهتر سیگنال ایجاد شده است. این پیشرفتها منجر به تغییراتی از قبیل\lr{IOT} \LTRfootnote{Internet of Things} و \lr{Pervasive Computing} در آینده ی نزدیک خواهد شد.
همچنین \lr{5G} منجر به توسعه و بهبود سرویسهای مخابراتی و اینترنتی سیار و در ورای آن، ایجاد تجربه‌ی بهتری برای مصرف کنندگان خواهد شد.\newline
برای توسعه‌ی اینترنت سیار و \lr{IOT}، نیازمند استفاده از شبکه‌ی نسل پنجم هستیم تا به سادگی منجر به دسترسی شبکه برای ارتباط انسان‌ها با یکدیگر و ارتباط ماشین با انسان گردد.

به طور کلی، 
\lr{5G}
 در سه نوع سرویس اصلی متصل از جمله پهن باند تلفن همراه، IoT عظیم و ارتباطات مهم برای ماموریت استفاده ‌‌‌‌‌.
\begin{enumerate}
\item 
پهن باند تلفن همراه پیشرفته 
\lr{(eMBB)}
 برای مقابله با نرخ داده‌های بسیار زیاد، تراکم بالای کاربران و ظرفیت ترافیک بسیار بالا برای سناریوهای مختلف و همچنین پوشش یکپارچه و سناریوهای تحرک بالا با نرخ داده‌های استفاده شده بهبود یافته است.
\item
 ارتباطات  عظیم ماشین 
 \lr{(mMTC)}
  برای
\lr{IoT}،
   برای تعداد بسیار زیاد دستگاه‌های متصل به مصرف کم و نرخ داده کم نیازمند می‌باشد.
\item 
ارتباطات بسیار مطمئن و با تأخیر کم
 \lr{(URLLC)}
 برای برنامه‌های کاربردی مهم برای ایمنی و ماموریت
  مورد توجه است.
\end{enumerate}
 از آنجا که ساختار
\lr{5G}
 کمتر به زیرساختهای \lr{4G} وابسته می‌شود و طیف بیشتری در دسترس قرار می‌دهد،
 تخمینها سرعت بارگیری را حداکثر 1000 برابر سریعتر از \lr{4G} در نظر دارد، که بالقوه از 
 \lr{10Gbps} 
بیشتر است، که به شما امکان می‌دهد تا در کمتر از یک ثانیه فیلم کامل
\lr{HD}
   را بارگیری کنید.
   برخی تخمینها محافظه کارانه تر هستند، اما حتی محافظه کارانه‌ترین تخمین نیز این نسل را چندین ده برابر سریعتر از \lr{4G} قرار می‌دهد.
دلایل نیاز به نسل پنجم اینترنت به طور خلاصه در ادامه بیان شده است \citep{etsi}.
\begin{itemize}
\item ترافیک داده‌های تلفن همراه به دلیل پخش ویدئو به سرعت، رو به افزایش است.
\item با در اختیار داشتن چندین دستگاه به طور همزمان، هر کاربر تعداد فزاینده‌ای از اتصالات را در اختیار دارد.
\item اینترنت اشیاء به شبکه‌هایی نیاز دارد که میلیاردها دستگاه را اداره کنند.
\item با وجود تعداد فزاینده‌ای از دستگاه‌های ارتباطی و افزایش ترافیک داده‌ها، هم دستگاه‌ها و هم شبکه‌ی  
آن
نیازمند افزایش بهره‌وری انرژی هستند.
\item 
 به دلیل تحت فشار قرار گرفتن اپراتورهای شبکه برای کاهش هزینه‌های عملیاتی و همچنین به دلیل اینکه کاربران به تعرفه‌های نرخ مسطح عادت می‌کنند و مایل نیستند مبلغ بیشتری بپردازند.
\item فناوری ارتباطات سیار میتواند موارد استفاده جدیدی را ایجاد کند (به عنوان مثال موارد تاخیر فوق العاده کم یا قابلیت اطمینان بالا) و برنامه‌های جدید برای صنعت که منجر به درآمدزایی بیشتر اپراتورها می‌گردد.
\end{itemize}
بنابراین عملکرد عملیاتی نسل پنجم می‌بایست به طور قابل توجهی افزایش یابد (به عنوان مثال افزایش راندمان طیفی، سرعت بالاتر داده، تأخیر کم).
زیرساخت
\lr{5G}
می‌بایست
در حالی که هنوز سطح قابل قبولی از مصرف انرژی، هزینه تجهیزات و استقرار شبکه و هزینه بهره برداری را ارائه می‌دهد،
 اینترنت اشیاء را به طور گسترده نیز تأمین کند.
 همچنین
از طیف گسترده‌ای از برنامه‌ها و خدمات پشتیبانی کند.
 \begin{figure}[H]
  \centering
    \includegraphics[width=0.8\textwidth]{./fig/etsi}
  \caption{مقایسه قابلیتهای کلیدی
\lr{IMT-Advanced}
 (نسل 4) با
\lr{IMT-2020}
 (نسل ۵) 
 با توجه به
\lr{ITU-R M.2083}
\cite{etsi}
  }
  \label{fig:C-RAN}
\end{figure}

 یکی از دلایل مهم رفتن محققان به سمت نسل پنجم، سرعت و نرخ انتقال بیشتری است که در ادامه به آن می‌پردازیم.
نیاز بشریت به ارتباط تلفنی (انتقال بدون سیم به صورت زمان حقیقی \LTRfootnote{Real Time} انسان را به سمت نسل اول ارتباطات \lr{1G} سوق داده است. نسل دوم ارتباطات \lr{2G} با سرویسهای انتقال پیام کوتاه ایجاد شد. همچنین با موفقیت تکنولوژی شبکه‌های منطقه ای بیسیم، اتصال به داده‌های اینترنتی مورد توجه عموم مردم قرار گرفت که پلی به سوی نسل سوم ارتباطات \lr{3G} را فراهم نمود. به طور منطقی پله‌ی‌ بعدی گام برداشتن در راستای کوچک شدن لپ تاپ و در آمیختن آن با تلفن که امروزه به صورت تلفن هوشمند\LTRfootnote{smart phone} است و دسترسی به اینترنت، پهنای باند بالا و داده‌ها در نقاط مختلف جهان بوده است که \lr{4G} یا نسل چهارم را به همراه داشته است.
با توجه به افزایش تعداد کاربران تلفن‌های
هوشمند و تبلت‌ها و افزایش نرخ ارسال اطلاعات و داده‌ها در طی سالهای اخیر طبق پیش بینی‌های سیسکو میزان ترافیک \lr{IP} طی سالهای اخیر
  چندین برابر افزایش خواهد یافت.
در نتیجه اپراتورها برای حل این مشکل و خدمات‌دهی بهتر ناچار به افزایش ظرفیت شبکه می‌باشند.
در ادامه به طور مختصر به نسلهای اخیر مخابراتی می‌پردازیم\cite{Gen}.
%\subsubsection{نسل اول}
%این اولین نسل از فناوری تلفن همراه بود. اولین نسل شبکه تلفن همراه تجاری در اواخر دهه 70 معرفی شد و استانداردهای کاملاً اجرا شده در دهه 80 تاسیس شد. استرالیا در سال 1987 توسط \lr{Telecom} (که امروزه با عنوان \lr{Telstra} شناخته می‌شود) معرفی شد، استرالیا اولین شبکه تلفن همراه خود را با استفاده از یک سیستم آنالوگ \lr{1G} دریافت کرد. \lr{1G} یک فناوری آنالوگ است و به طور کلی تلفنها از باتری ضعیف برخوردار هستند و کیفیت صدا بدون امنیت بسیار زیاد بود و گاهی اوقات تماسهای کاهش یافته را تجربه میکنید. این استانداردهای ارتباطی آنالوگ ارتباطی است که در دهه 1980 معرفی شد و تا زمانی که جایگزین ارتباطات دیجیتال \lr{2G} شود، ادامه یافت. حداکثر سرعت \lr{1G} 2.4 کیلوبیت بر ثانیه است.
%\subsubsection{نسل دوم}
%تلفنهای همراه وقتی از \lr{1G} به \lr{2G} رفتند، اولین نسخه اصلی خود را دریافت کردند. تفاوت اصلی بین دو سیستم تلفن همراه (\lr{1G} و \lr{2G}) در این است که سیگنالهای رادیویی مورد استفاده شبکه \lr{1G} آنالوگ هستند، در حالی که شبکه‌های 
%\lr{2G}
% دیجیتال هستند. انگیزه اصلی این نسل، تهیه کانال ارتباطی ایمن و مطمئن بود. این نسل همچنین مفهوم \lr{CDMA} و \lr{GSM} را پیاده سازی کرد که خدماتی مانند پیامک را ارائه داده است. شبکه‌های مخابراتی سلولی نسل دوم بطور تجاری در سال 1991 توسط رادیولینجا (در حال حاضر بخشی از الیزا اویج) توسط استاندارد \lr{GSM} در فنلاند راه اندازی شد.
%\lr{2G}
%قابلیتهایی از قبیل
%\lr{multiplexing}
%دارد که 
%به چندین کاربر در یک کانال منفرد 
%اجازه ی انتقال داده می‌دهد.
%انتقال داده و صدا در این نسل وجود داشته است.
%در این نسل مخابرات، 
%سرویسهای اساسی مانند پیام کوتاه، رومینگ داخلی، تماسهای کنفرانسی، نگه داشتن تماس و صورتحساب مبتنی بر خدمات معرفی شده است. جداسازی هزینه‌های مبتنی بر تماسهای مسافت طولانی و صورتحساب در زمان واقعی نیز از قابلیتهای این نسل بود.
%حداکثر سرعت برای سرویس \lr{GPRS}
%\LTRfootnote{General Packet Radio Service}
%\lr{50Kbps}
%و برای سرویس 
%\lr{EDGE}
%\LTRfootnote{Enhanced Data Rates for GSM Evolution}
%\lr{1Mbps}
%می‌باشد.
%\subsubsection{نسل سوم}
%این نسل مخابراتی استانداردهای بسیاری از فناوریهای بیسیم را تعیین نمود. مرورگر وب، ایمیل، بارگیری ویدیو، به اشتراک گذاری عکس و سایر فناوریهای هوشمند در نسل سوم معرفی شدند که در سال 2001 به طور تجاری معرفی شد. اهداف تعیین شده برای ارتباطات سیار نسل سوم، تسهیل ظرفیت بیشتر صدا و داده، پشتیبانی از طیف گسترده تری از برنامه‌ها و افزایش انتقال داده با هزینه کمتری بود.
%استاندارد \lr{3G} از فناوری جدیدی به نام
% \lr{UMTS}
% \LTRfootnote{Universal Telecommunications System Mobile}
% به عنوان معماری اصلی شبکه خود 
%استفاده می‌کند.
%\lr{3G}
% دارای پشتیبانی خدمات چندرسانه ای است.
%\lr{3G}
%  با بهبود چگونگی فشرده سازی صدا در طی تماس، راندمان طیف فرکانس را افزایش داده است، بنابراین تماس همزمان بیشتری میتواند در همان محدوده فرکانس اتفاق بیفتد.
%حداکثر سرعت این نسل 
%\lr{2Mbps}
%و
%حداکثر سرعت نظری برای HSPA + 21.6 Mbps است. 
در ادامه مروری بر نسلهای مختلف مخابرات خواهیم داشت.
\subsection{ نسل چهارم مخابرات}
\lr{4G}
 یک فناوری بسیار متفاوت در مقایسه با \lr{3G} است و هدف از آن، فراهم آوردن سرعت بالا، کیفیت بالا و ظرفیت بالا برای کاربران در عین بهبود امنیت و کاهش هزینه خدمات صوتی و دیتا، چندرسانه ای و اینترنت از طریق \lr{IP} می‌باشد. برنامه‌های کاربردی بالقوه و جاری شامل دسترسی به وب موبایل اصلاح شده، تلفن تلفنی \lr{IP}، خدمات بازی، تلویزیون همراه با کیفیت بالا، کنفرانس ویدیویی، تلویزیون سه بعدی و محاسبات ابری از قابلیتهای پشتیبانی آن می‌باشد.

فن آوریهای کلیدی که این امکان را ایجاد کرده اند 
\lr{MIMO} \LTRfootnote{Multiple Output Multiple Output}
 و
  \lr{OFDM} \LTRfootnote{Multiplexing Division Frequency Division}
می‌باشد.
دو استاندارد مهم آن
\lr{LTE} \LTRfootnote{Long Term Evolution}
و
\lr{WiMAX}
می‌باشد.
حداکثر سرعت یک شبکه 4G هنگام حرکت دستگاه 100 مگابیت بر ثانیه یا 1 گیگابیت بر ثانیه برای ارتباطات کم تحرک مانند هنگام ایستادن یا راه رفتن است. تأخیر از حدود \lr{300ms} به \lr{100ms} با کاهش تراکم دست می‌یابد.
 \subsection{نسل پنجم مخابرات}
تکنولوژی 5G یک استاندارد صنعتی است که جایگزین استاندارد رایج کنونی یعنی 4G LTE خواهد شد. این فناوری پنجمین نسل از استاندارد سلولی است. طراحی این استاندارد به گونه‌ای است که سرعت آن از تکنولوژی 4G LTE بسیار سریع تر است. البته هدف این استاندارد صرفا افزایش سرعت اتصالات اینترنتی تلفنهای هوشمند نیست. این استاندارد، اینترنت بی سیم بسیار پر سرعتی را در همه جا و برای همه چیزها از جمله خودروهای متصل، خانه‌های هوشمند و ابزارهای اینترنت اشیا (IoT)، فراهم خواهد کرد. 
کاهش مصرف انٰرژی  معیاری است که در این نسل به آن توجه شده‌است و دستگاه‌های فرستنده و گیرنده اپراتورها باید در ساعت کم مصرف به حالت صرفه‌جویی انرژی وارد شده و به سرعت فعال شوند که این معیار در نسل چهارم قید نشده بوده‌است.
 با توجه
به این که نرخ داده و ظرفیت در سیستمهای نسل چهارم به ظرفیت
شانون نزدیک شده است، در نتیجه روشهایی که برای
افزایش ظرفیت شبکه مورد استفاده میگیرند که به شرح زیر است:
\begin{itemize}
\item
استفاده از تکنیک \lr{Massive Mimo}
\item
استفاده از روشهای پردازشهای ابری
\item
شبکه ی تعریف شده ی نرم‌افزاری
\lr{SDN}
\LTRfootnote{Software Defined Networking}
\item
موج میلیمتری
\LTRfootnote{mm Wave}
\item 
ساختار شبکه‌های دسترسی رادیویی باز
\lr{ORAN}
\LTRfootnote{Open Radio Access Network}
\item 
مجازی سازی توابع شبکه
\lr{NFV}
\LTRfootnote{Network Function Virtualization}
\item 
برش شبکه
\LTRfootnote{Network Slicing}
\end{itemize}
\subsection{نسل ششم مخابرات}
نسل ششم شبکه‌های مخابراتی 
(\lr{6G})
ظرفیت شبکه را تا 
$10\text{Gbps/m}^{3}$
افزایش داده است. همچنین، این نسل تاخیر انتها به انتهای زیر 
$1\text{ms}$
و نرخ انتقال داده ی بالای 
$1 \text{Tbps}$ 
را در نظر گرفته است.
قابلیت‌های \lr{6G} برنامه‌ها و سرویس‌های جدیدی از جمله ارتباطات هولوگرافیک، تعامل بی‌سیم مغز و ماشین، رانندگی خودکار و غیره را باز می‌کند.
~\cite{slawomir20216g}.

در حوزه ارتباطات بی سیم، \lr{6G}  نشان دهنده تغییر پارادایم بعدی است که چندین ویژگی و پیشرفت جدید را در مقایسه با نسل های قبلی خود معرفی می کند. این بخش پیشرفت‌ها و نوآوری‌های کلیدی پیش‌بینی‌شده در شبکه‌های \lr{6G}  را با تکیه بر تحقیقات دانشگاهی و بینش‌های متخصص مورد بحث قرار می‌دهد.
\begin{itemize}
	\item
	 ارتباط تراهرتز (\lr{THz}): یکی از پیشرفت‌های اولیه در \lr{6G} استفاده از فرکانس‌های تراهرتز برای ارتباطات بی‌سیم است. امواج تراهرتز در مقایسه با فرکانس‌های امواج مایکروویو و میلی‌متری استفاده شده در نسل‌های قبلی، پهنای باند بسیار بالاتری را ارائه می‌دهند. این امر امکان افزایش مرتبه‌ای در نرخ داده‌ها را فراهم می‌آورد و فرصت‌های جدیدی را برای برنامه‌های کاربردی با پهنای باند فشرده مانند پخش ویدئو با کیفیت فوق‌العاده، ارتباطات هولوگرافیک و تجربه‌های واقعیت مجازی فراگیر باز می‌کند.
	\item 
	\lr{extreme-MIMO}:
	یکی از مهم‌ترین تغییرات در نسل ششم، استفاده از تعداد آنتهای بسیار زیاد در ورودی و خروجی می‌باشد.
	در این نسل مخابرات هدف قرار دادن ۱۰۲۴ المان آنتن در واحدهای رادیویی می‌باشد. پهنای باند در این حالت از 
	$100 \text{MHz}$
	به 
	$400 \text{MHz}$
	می‌رسد و  تعداد فرستنده و گیرنده به ۵۱۲ تا ارتقا می‌یابد.
	\item
	\lr{FR3}:
	باند جدید فرکانسی \lr{FR3}
	که شامل باند فرکانسی 
	$7-15 \text{GHz}$
	می‌باشد.
	\item 
	شبکه‌های مجهز به هوش مصنوعی: انتظار می‌رود تکنیک‌های هوش مصنوعی (\lr{AI}) و یادگیری ماشین نقش مهمی در شبکه‌های \lr{6G}  ایفا کنند. هوش مصنوعی را می توان برای کارهای مختلفی مانند تخصیص منابع هوشمند، بهینه سازی شبکه، مدیریت تداخل و تجزیه و تحلیل پیش بینی کننده استفاده کرد. با به کارگیری الگوریتم‌های هوش مصنوعی، شبکه‌های \lr{6G}  می‌توانند با محیط‌های پویا و پیچیده سازگار شوند، عملکرد سیستم را بهینه کنند و خدمات شخصی‌سازی شده را متناسب با نیازهای کاربر ارائه دهند.
	\item 
	ارتباطات و امنیت کوانتومی: پیش‌بینی می‌شود که ارتباطات کوانتومی و رمزنگاری اجزای جدایی‌ناپذیر شبکه‌های \lr{6G}  باشند و نگرانی‌های امنیتی در حال رشد در عصر دیجیتال را برطرف کنند. ارتباطات کوانتومی از اصول مکانیک کوانتومی برای اطمینان از انتقال ایمن اطلاعات، ارائه سطوح بی سابقه ای از رمزگذاری و محافظت در برابر استراق سمع استفاده می کند. ترکیب فناوری‌های کوانتومی در شبکه‌های \lr{6G}  امنیت و حریم خصوصی داده‌های کاربر و کانال‌های ارتباطی را افزایش می‌دهد.
\end{itemize}
\section{مقدمه ای بر ساختار \lr{ORAN}}
مجازی سازی \lr{RAN} توجه زیادی را از طرف اپراتورها به خود جلب می‌کند، زیرا منجر به کاهش هزینه‌های اپراتور و \lr{opex} می‌شود و همچنین این امکان را برای آنها فراهم کرده تا با سرعت بیشتری قابلیتهای جدیدی به شبکه اضافه کنند.

این احتمال وجود دارد که همه این علاقه‌ها در ایجاد سه گروه مختلف باشد -
 انجمن 
 \lr{xRAN} 
، گروه
 \lr{ OpenRAN }
  شرکت 
 \lr{ Telecom Infra}
   و ابتکار عمل
\lr{Open VRAN}
که برای شرکت سیسکو می‌باشد.
اگرچه همه این گروه‌ها می‌گویند که در حال کار بر روی یک چیز هستند، که اساساً برای باز کردن \lr{RAN} با استفاده از رابطهای استاندارد و عناصر شبکه جعبه سفید است، اما در بررسی دقیق تر اختلافاتی نیز وجود دارد.

شبکه‌ی دسترسی باز 
\LTRfootnote{Open RAN}(\lr{ORAN})
تبسیط و ترکیبی از دو ساختار \lr{C-RAN} \LTRfootnote{Cloud Radio Access Network} و \lr{xRAN} می‌باشد که انتظار می‌رود که در فناوری نسل پنجم مخابرات مورد استفاده قرار گرفته و منجر به بهبود عملکرد شبکه‌های دسترسی رادیویی \lr{RAN} گردد. 
این ساختار یک شبکه ی باز، انعطاف پذیر و هوشمند است.


\lr{ORAN} 
توابع
 شبکه‌ی دسترسی رادیویی 
 را به سه قسمت تقسیم می‌کند،
  که قسمت اول واحد از راه دور
 \lr{(RU)} \LTRfootnote{remote unit} 
، واحد توزیع شده
  \lr{(DU)} \LTRfootnote{Distributed unit} 
  و واحد مرکزی 
   \lr{(CU)} \LTRfootnote{Central unit} 
   می‌باشد.
   در حالی که \lr{RU} دارای توابع فیزیکی \LTRfootnote{Physical layer} \lr{(PHY)} لایه‌ی پایین تر است،
    \lr{DU} حاوی \lr{(PHY)} بالاتر، 
    \lr{MAC} 
 \LTRfootnote{Medium Access Control}   
    و
     \lr{RLC}
   \LTRfootnote{Radio Link Control}  
      است     
    و 
    \lr{(CU)}
     حاوی
     \lr{RRC}
     \LTRfootnote{Radio Resource Control}
     ،\lr{PDCP}
      \LTRfootnote{Packet Data Convergence Protocol}
      و 
      \lr{SDAP}
      \LTRfootnote{Service Data Adaptation Protocol}
      است.
      
\lr{DU}
و
\lr{CU}
به عنوان توابع شبکه مجازی \lr{(VNFs)} پیاده سازی می‌شوند،
که در یک محیط ابر اجرا می‌شود.

رابطهای بین \lr{RU}، \lr{CU} و \lr{DU} رابطهای استاندارد باز هستند.
\subsection{مقدمه ای بر ساختار شبکه‌های دسترسی رادیویی \lr{C-RAN}}
شبکه‌های دسترسی رادیویی‌ ابری منجر به افزایش پوشش ارسالی می‌گردد. با توجه به ساختار شبکه
   \lr{C-RAN} که معماری جدیدی را برای شبکه‌های نسل آینده
ارائه می‌دهد، نه تنها ظرفیت شبکه افزایش می‌یابد بلکه
مشکلاتی که در روشهای دیگر وجود دارد را نیز هموار
می‌سازد.
مفهوم شبکه دسترسی رادیو ابر \lr{C-RAN}، به مجازی سازی کارکردهای ایستگاه  پایه \LTRfootnote{Base Station-BS} با استفاده از تکنولوژی رایانش ابری \LTRfootnote{Cloud Computing} اشاره می‌نماید. این مفهوم به ایجاد یک ساختار سلولی جدید منجر می‌شود که در آن، نقاط دسترسی بیسیم کم هزینه که با عنوان واحدهای رادیویی \LTRfootnote{Radio Units} و یا رادیو هدهای
  راه دور 
  \LTRfootnote{Radio Remote Heads}
 شناخته می‌شوند- با استفاده از یک ابر متمرکز با قابلیت پیکربندی مجدد و یا واحد مرکزی \LTRfootnote{Control Unit} مدیریت می‌شوند. شبکه امکان کاهش هزینه‌های سرمایه گذاری و عملیاتی مورد نیاز برای اپراتورها به منظور توسعه و نگهداری شبکه‌های ناهمگن متراکم را فراهم می‌آورد. این مزیت مهم در کنار بازده طیفی، تسهیم آماری \LTRfootnote{Statisitical Multiplexing}، و مزیتهای متعادل سازی بار باعث می‌شود تا شبکه \lr{C-RAN} به عنوان یکی از تکنولوژیهای کلیدی در توسعه سیستمهای \lr{5G} در جایگاه بسیار مناسبی قرار بگیرد. در ادامه، یک بررسی کلی و مختصر از تحقیقات جدید در مورد ساختار \lr{C-RAN} ارائه می‌شود و موضوعات مورد تاکید عبارتند از فشرده سازی لینک \lr{fronthaul} پردازش باند پایه، کنترل دسترسی به محیط واسط، تخصیص منابع، ملاحظات سطح سیستم، و تلاشهای انجام شده در راستای ارائه استانداردها.
\subsubsection{ساختار شبکه‌های مختلف }
با توجه به مقاله ی\cite{checko2015cloud}،
\begin{figure}
  \centering
    \includegraphics[scale=0.7]{./fig/c11}
  \caption{ساختار سنتی ایستگاه پایه \cite{checko2015cloud}}
  \label{fig:c11}
\end{figure}
هر ایستگاه پایه دو نوع پردازش انجام می‌دهد : پردازش
رادیویی که توسط واحد رادیویی \LTRfootnote{RRH} انجام می‌شود و شامل پردازش
دیجیتالی، فیلترینگ فرکانسی، تقویت توان و ....می‌باشد و
پردازش باند پایه که توسط واحد باند پایه \LTRfootnote{BBU} که همان واحد کنترل است \LTRfootnote{CU} انجام شده و از جمله
مهمترین وظایف آن می‌توان به کدینگ، مدولاسیون و
تبدیل فوریه ی سریع اشاره کرد. در ساختار جدیدی که
تحت عنوان \lr{C-RAN}  معرفی خواهیم نمود نحوه ی ارتباط
پردازشگرهای رادیویی و باند پایه متحول شده و در نتیجه
مزایایی برای شبکه حاصل خواهد شد.در ادامه، انواع ساختارها را بیان خواهد شد.
\subsubsection{ساختار سنتی ایستگاه پایه }

در ساختارهای سنتی ایستگاه پایه، پردازشهای رادیویی و باند پایه در
داخل ایستگاه پایه انجام ‌شد و مدول آنتن نیز در فاصله
ی چند متری از مدول رادیویی نصب شده و ارتباط آنها
توسط کابل کواکسیال برقرار میشد که همین امر سبب
افزایش تلفات در شبکه می‌باشد. این نوع ساختار در شکل
\ref{fig:c11} نشان داده شده است. همان گونه که مشاهده میکنید
ارتباط بین ایستگاه‌های پایه توسط ارتباط  $X_2$ و ارتباط بین
ایستگاه پایه و شبکه ی هسته توسط ارتباط $ S_1$ برقرار می
شود. این نوع ساختار در شبکه‌های \lr{1G} و \lr{2G} به کار گرفته
شده است 
\cite{checko2015cloud}.

%Figure \ref{fig:gull} shows a photograph of a gull.
\subsubsection{ ساختار ایستگاه پایه و واحد رادیویی}
\begin{figure}
  \centering
    \includegraphics[scale=0.7]{./fig/c22}
  \caption{ ساختار ایستگاه پایه و واحد رادیویی \cite{checko2015cloud}}
  \label{fig:c22}
\end{figure}
در این ساختار واحد رادیویی و واحد پردازشی سیگنال، از هم
مجزا شده و واحد رادیویی که تحت عنوان \lr{RRH} یا \lr{RRU}
نیز شناخته می‌شود، توسط فیبر نوری به واحد باند پایه یا \lr{BBU} اتصال می‌یابد. همان طور که پیشتر بیان شد واحد رادیویی مسئولیت
انجام پردازشهای دیجیتالی از جمله تبدیل انالوگ به
دیجیتال، دیجیتال به انالوگ، تقویت توان و فیلترینگ رابر عهده دارد، که تفکیک وظایف واحد پردازشی و واحد
رادیویی در این ساختار در شکل \ref{fig:c22} قابل مشاهده است. این
نوع ساختار برای شبکه‌های نسل سوم معرفی شده و امروزه
نیز بیشتر ایستگاه‌های پایه از همین ساختار بهره میگیرند.
از جمله ویژگیهای بارز این ساختار امکان ایجاد فاصله
بین واحد رادیویی و پردازشی می‌باشد، که این فاصله به
دلیل تاخیر پردازشی و انتشاری نمیتواند از  $40$کیلومتر
فراتر رود. در این ساختار تجهیزات مرتبط با \lr{BBU} می
توانند به مکانی مناسبتر که قابل دسترس تر بوده و هزینه
ی اجاره و نگهداری کمتری را به اپراتورها تحمیل می
کنند منتقل شوند و واحدهای رادیویی نیز در در پشت بام
ساختمانها و مکانهای مرتفع نصب می‌شوند که این
خود سبب کاهش هزینه‌های خنک سازی ادوات موجود
می‌شود. نحوه ی ارتباط بین \lr{RRH} و \lr{BBU} مشابه ساختار
سنتی بوده و \lr{RRH}ها نیز توسط معماری زنجیروار باهم
در ارتباطند.
\begin{figure}
  \centering
    \includegraphics[scale=0.7]{./fig/c33}
  \caption{ساختار  \lr{C-RAN} \cite{checko2015cloud}}
  \label{fig:c33}
\end{figure}
\subsubsection{ساختار \lr{C-RAN}}
در ادامه ساختارهای شبکه دسترسی رادیویی ابری و ساختارهای بهبود یافته ی آن را معرفی می‌نماییم.
\begin{itemize}
\item \textbf{شبکه‌های دسترسی رادیویی ابری}


ایده اصلی \lr{C-RAN} جداسازی بخش رادیویی (\lr{RRH}) 
\LTRfootnote{Radio Remote Head}
 از واحد پردازشی باند پایه (\lr{BBU})
 \LTRfootnote{Baseband Unit}
  است.
از تجمیع \lr{BBU}ها بر روی سرور ابری، \lr{BBU-Pool} ایجاد می‌شود.
در این ساختار، در راستای بهینه سازی عملکرد \lr{BBU}
ها در مواجهه باایستگاه‌های پایه پر ترافیک و کم ترافیک،
 \lr{BBU}ها به صورت یک مجموعه ی واحد تحت عنوان 
\lr{BBU Pool}
 در آمده اند که این مجموعه بین چندین سلول 
 به اشتراک گزارده شده و مطابق شکل زیر مجازی سازی
می‌شود. 
در توضیح بیشتر این ساختار میتوان این گونه
عنوان کرد که \lr{BBU Pool} به عنوان یک خوشه ی مجازی
در نظر گرفته می‌شود که شامل پردازش گرهایی می‌باشد
که پردازش‌‌ باند پایه را انجام می‌دهند. ارتباط بین
  \lr{BBU}ها در ساختارهای فعلی به شکل  $X_2$ برقرار می‌شود
که در این ساختار ارتباط بین خوشه‌ها از فرم جدید $X_2$
تحت عنوان  $X_2 +$برقرار می‌شود.
\newline
در شکل \ref{fig:C-RAN} ساختار کلی شبکه‌ی  \lr{C-RAN} در سیستمهای
\lr{ LTE}
 نمایش داده شده است. همان طور که در شکل قابل
مشاهده می‌باشد ساختار کلی شبکه  \lr{C-RAN} به دو بخش
 \lr{backhaul} و \lr{fronthaul} تقسیم بندی شده‌است. بخش
 \lr{fronthaul}شبکه به مرحله ی اتصال سایتهای \lr{ RRH}به
 به \lr{BBU Pool} به اتصال \lr{backhaul} و بخش \lr{BBU Pool}
هسته ی شبکه ی سیار اطلاق می‌شود. همان گونه که قبلا
ذکر شد \lr{ RRH}ها در نزدیکی انتن نصب شده و از طریق
لینکهای انتقالی نوری با پهنای باند وسیع و تاخیر کم به
پردازشگرهای قوی در \lr{BBU} متصل می‌شوند. توسط این
لینکهای انتقالی است که سیگنالهای دیجیتالی باند
پایه از نوع \lr{IQ} بین \lr{RRH} و \lr{BBU} انتقال می‌یابند \cite{checko2015cloud}.
\begin{figure}
  \centering
    \includegraphics[width=\textwidth]{./fig/CRAN}
  \caption{ساختار شبکه‌ی \lr{C-RAN} \cite{checko2015cloud}}
  \label{fig:C-RAN}
\end{figure}
\item \textbf{شبکه‌های دسترسی رادیویی ابری نامتجانس (\lr{H-CRAN})}


برای غلبه بر چالشهای شبکه‌های \lr{C-RAN} با محدودیتهای \lr{fronthaul}، شبکه‌های دسترسی ابری نامتجانس (\lr{H-CRAN}) معرفی می‌شود\cite{ fogComputing, heterogeneous, fogEdge}.
\begin{figure}
  \centering
    \includegraphics[scale = 0.8]{./fig/hc}
  \caption{ ساختار شبکه‌های دسترسی ابری نامتحانس \cite{heterogeneous}  }
  \label{fig:hc}
\end{figure}

صفحه‌ی کاربر و صفحه ی کنترلگر در چنین شبکه‌هایی از هم مجزا می‌باشند. 
در این شبکه‌ها، نودهای توان بالا   \LTRfootnote{High Power Node}\lr{HPN}، عمدتا برای فراهم کردن پوشش بدون درز و اجرای عملکرد صفحه کنترل می‌باشد. در حالی که \lr{RRH}ها برای فراهم نمودن سرعت بالای نرخ داده برای انتقال بسته در ترافیک قرار گرفته اند. \lr{HPN}ها از طریق لینکهای \lr{backhaul}  به \lr{BBU Pool} متصلند ( برای هماهنگ کردن تداخل ).\newline
ساختار این شبکه شبیه به ساختار \lr{C-RAN} می‌باشد. همانطور که در شکل \eqref{fig:hc} نشان داده شده است، تعداد زیادی \lr{RRH}، همراه با انرژی مصرفی کم در ساختار \lr{H-CRAN}، با یکدیگر در \lr{BBU Pool} مرکزی، همکاری می‌کنند تا گین مشترک بالایی بدست آورند. تنها‌، فرکانس رادیویی جلو، (\lr{RF}) و عملکردهای پردازشی  ساده، در \lr{RRH}، صورت میگیرد، در حالی که پردازشهای مهم دیگر، در \lr{BBU Pool} انجام می‌گیرد. همچنین تنها بخشی از عملکردها در لایه ی \lr{PHY} در \lr{RRH} به مشارکت می‌انجامد که این مدل در شکل \eqref{fig:hc} نشان داده شده است.\newline
اگرچه، برخلاف \lr{C-RAN}،
\lr{BBU Pool} در \lr{H-CRAN}، به \lr{HPN}ها متصلند که این، برای کاهش تداخل متقابل بین \lr{RRH}ها و \lr{HPN}ها از طریق محاسبات ابری متمرکز براساس تکنیکهای پردازشی مشترک می‌باشد. همچنین، داده و واسط کنترل، بین \lr{BBU Pool} و \lr{HPN}های $S_1$ و $X_2$ شناخته شده اند که تعریف آنها بر اساس تعریف استاندارد \lr{3G} ایجاد شده است.\newline
همانطور که سرویسهای صدا، میتوانند به صورت بهینه در طول مد سوییچ بسته در \lr{4G} فراهم گردند، \lr{H-CRAN} میتواند به طور همزمان سرویس صدا و داده را پشتیبانی کند. سرویس صدا مرجح به اداره از طریق \lr{HPN}ها می‌باشد، در حالی که ترافیک بسته ی پر داده، بیشتر توسط \lr{RRH} اداره می‌گردد. 
در مقایسه با ساختار \lr{C-RAN}، ساختار \lr{H-CRAN} نیازهای \lr{fronthaul} را بوسیله ی مشارکت \lr{HPN}ها برطرف می‌سازد. با توجه به حضور \lr{HPN}ها، سیگنالهای کنترلی و سمبلهای داده در \lr{H-CRAN} جدا از هم می‌باشند. تمام کنترل کننده‌های سیگنال و سیستمهایی که اطلاعات را ارسال می‌نمایند، توسط \lr{HPN}ها به \lr{UE}، منتقل می‌گردد که منجر به سادگی در ظرفیت و در محدودیت تاخیر زمان در لینکهای \lr{fronthaul } بین \lr{RRH}ها و \lr{BBU Pool}  می‌گردد و منجر به صرفه‌جویی در مصرف انرژی می‌گردد. همچنین، برخی از ترافیکهای شدید و ناگهانی \LTRfootnote{Burst Traffic} و یا سرویس پیام همراه با مقدار داده‌ی کم، می‌تواند به صورت بهینه توسط \lr{HPN}ها پشتیبانی گردد. مکانیزم کنترل بین ارتباط داشتن و نبود ارتباط، توسط \lr{H-CRAN} پشتیبانی می‌گردد که منجر به حفظ کردن مقدار قابل توجهی \lr{Overhead} در رادیو بوسیله ی مکانیزم ارتباط جهت دار خالص می‌گردد. در \lr{RRH}، تکنولوژیهای مختلف انتقال در لایه ی \lr{PHY}، قابل استفاده برای بهبود نرخ انتقال (همانند موج میلیمتری و نور مرئی) می‌گردد. در \lr{HPN}ها، \lr{MIMO}\LTRfootnote{Multiple Input Multiple Output}، یکی از راه‌های افزایش پوشش در بهبود ظرفیت می‌باشد.

\item \textbf{ساختار دسترسی رادیویی مهی}


برای حل کردن مشکلات \lr{H-CRAN} و \lr{C-RAN}، نیاز به معرفی ساختار جدید دیگری می‌باشیم که آن را \lr{F-RAN} می‌نامیم.
\lr{F-RAN} 
تمام ویژگیهای مثبت محاسبات ابری و شبکه‌های نامتجانس و محاسبات مهی را همزمان در بر می‌گیرد.
محاسبات مهی، اصطلاحی برای جایگزین کردن محاسبات ابری است که مقدار قابل توجهی از ذخیره سازی، ارتباطات، کنترل کردن، اندازه گیری و مدیریت را در لبه ی شبکه انجام می‌دهد (نه در کانال و ابر مرکزی) \cite{fogComputing, fogEdge}.
 سیستمهای \lr{F-RAN} تحولی از سیستمهای \lr{C-RAN} می‌باشد که برخی از ارتباطات توزیع شده و عملکردهای ذخیره سازی در منطق لایه ی مه قرار دارد. همچنین چهار نوع ارتباطات ابری تعریف شده است.
  \begin{figure}[H]
  \centering
    \includegraphics[scale =0.7]{./fig/fr}
  \caption{ مدل سیستم \lr{F-RAN} \cite{fogComputing} }
  \label{fig:fr}
\end{figure}
 \begin{itemize}
 \item
 ابر ذخیره‌گر و ارتباطات مرکزی جامع :
 که همانند ابر مرکزی \lr{C-RAN} می‌باشد.
 \item
 ابر کنترل‌گر مرکزی :که برای تکمیل عملکردهای کنترلی می‌باشد و در \lr{HPN}ها قرار دارد.
 \item
 ابر ارتباطات منطقی توزیع شده که در برنامه‌های محاسبات مهی و ابزارهای این محاسبات قرار دارد.
 \item
  ابر ذخیره گر منطق توزیع شده:
  که همانند قبل در \lr{F-RAN} قرار دارد.
 \end{itemize}
 در این ساختار، برای کاهش تاخیر ناشی از انتقال داده‌ها به ابر مرکزی، ساختارهای \lr{RRH} را دارای حافظه قرار می‌دهیم که برای ارتباطات محلی، به جای اینکه پردازشها در \lr{BBU Pool} صورت بگیرد، بدون نیاز به انتقال به ابر مرکزی، درون \lr{RRH}ها انجام پذیرد. 
\end{itemize}
\subsection{\lr{xRAN}}
\lr{xRAN}
در سال ۲۰۱۶ با هدف استانداردسازی یک جایگزین انعطاف پذیر و باز برای \lr{RAN}
مبتنی بر سخت افزار سنتی بدست آمده‌است.
 در این ساختار، سه حوزه ی مهم مورد بررسی قرار گرفته است.
اولین حوزه ی مورد بررسی، جداسازی بخش
صفحه ی کنترل
 \LTRfootnote{control plane} از 
 صفحه‌ی کاربر
\LTRfootnote{user plane}
می‌باشد. حوزه ی دوم،
ساختن یک پشته نرم‌افزاری \lr{eNodeB} مدولار که از سخت افزار \lr{COTS} استفاده می‌کند، می‌باشد.
حوزه ی سوم مورد بررسی، انتشار رابطهای باز شمال و جنوب است\cite{xran}.
در ادامه این سه حوزه به طور دقیق تر مورد بررسی قرار میگیرد\cite{xran1}.

\begin{itemize}
\item \textbf{ جداسازی بخش صفحه ی کنترل از 
صفحه ی کاربر}:
این انتقال صفحه ی کنترل، که قبلاً کاملاً به دستگاه‌های سخت افزاری \lr{RAN} متصل بود، به دستگاه‌های محاسباتی در دسترس امکان می‌دهد \lr{RAN} بتواند به عنوان یک استخر منطقی از ظرفیت، با کارایی بیشتری کار کند.
نرم‌افزار \lr{eNodeB} از سخت افزار خاص فروشنده جدا می‌شود و الهام بخش نوآوری در هر دو نرم‌افزار و سخت افزار به صورت مشارکتی اما به طور مستقل است.
برنامه نویسی و کنترل زمان واقعی بی سابقه در زیرساختهای \lr{RAN} به دست آمده است، که به راحتی از برنامه‌های کاربردی تلفن همراه و خدمات تجاری پشتیبانی می‌کند.
\item \textbf{ساختن یک پشته نرم‌افزاری \lr{eNodeB} مدولار}:
رویکرد \lr{xRAN} به خوبی با طرحهای مجازی سازی عملکرد شبکه حامل \lr{(NFV)} مطابقت دارد، و همچنین منجر به کنترل عملکرد ترافیک با کارایی بالا، مدیریت تداخل و کنترل منابع رادیویی روی سیستم عاملهای استاندارد \lr{x86} می‌شود.
\item \textbf{انتشار رابطهای باز شمال و جنوب}: 
رابطهای استاندارد و باز قابلیت پشتیبانی از فروشنده‌های متعدد همکاری اثبات شده دارند. 
\lr{xRAN.org}
و اعضای آن به تصویب رساندن این رابطها از طریق فرآیندهای استاندارد منجر به در دسترس قرار دادن معماری \lr{xRAN} و پشتیبانی مورد نیاز می‌شوند.

\end{itemize}
در ادامه مزایای ساختار xRAN را بیان می‌نماییم.

\textbf{مزایای ساختار xRAN}
%\subsubsection{مزایای ساختار \lr{xRAN}}
\begin{itemize}
\item 
جداسازی بخش صفحه ی کنترل از 
صفحه ی کاربر 
منجر به 
برنامه ریزی زمان واقعی بی سابقه و کنترل در زیرساخت \lr{RAN} می‌شود که به راحتی برنامه‌های کاربردی تلفن همراه و خدمات تجاری را پشتیبانی می‌کند.
\item
 یک پشته
 \lr{eNB} 
 مدولار مبتنی بر نرم‌افزار، منجر به امکان قرارگیری انعطاف پذیر توابع \lr{eNB} و کنترل ترکیبی آن با یک برنامه ریز امکان پذیر می‌شود تا بتواند زمان تاخیر متغیر در \lr{fronthaul} را کنترل کند.
\item
رابط‌های مرزی جنوبی استاندارد، پیاده سازی شبکه با خرید سیستم از چندین شرکت متفاوت را امکان پذیر می‌سازد و رابطهای شمال مرزی، برش کامل شبکه برای بهینه سازی \lr{QoE} \LTRfootnote{Quality of Experience} کاربر را فراهم می‌کند.
رابطهای \lr{xRAN} به خوبی با لبه ابر حامل هماهنگ هستند و اجازه می‌دهد تا محاسبه و ذخیره سازی منابع در شبکه تلفن همراه 
به صورت دینامیکی مدیریت شود.
\item 
این ساختار هزینه‌ی رشد ظرفیت دسترسی رادیویی و هزینه‌ی بهره برداری را کاهش می‌دهد.
\end{itemize}
\subsection{vRAN}
vRAN 
یا شبکه‌های دسترسی رادیویی مجازی
گونه‌ی دیگری از شبکه‌های رادیویی دسترسی می‌باشند که منجر به افزایش هوشمندانه ظرفیت، کاهش چشمگیر هزینه‌ها می‌شود. همچنین قابلیت انعطاف پذیری و مقیاس پذیری پویا را فراهم می‌کند که برای پشتیبانی از خدمات و برنامه‌های آینده ضروری خواهد بود.
معماری vRAN با اجرای توابع باند پایه مجازی بر روی سخت افزار سرور کالا، بر اساس اصول مجازی سازی توابع شبکه (NFV)، فراتر از آخرین شبکه‌ی  متمرکز رادیویی (C-RAN) است.
معماری C-RAN می‌تواند با ایجاد امکان تجمع منابع پردازش باند پایه، که می‌تواند به صورت پویا به سایتهای مختلف سلول و فن‌آوریهای رادیویی اختصاص یابد، گامی فراتر رود.
به اشتراک گذاری منابع باند پایه از طیف موجود با کارآیی بیشتری استفاده می‌کند و قابلیت اطمینان سرویس را بهبود می‌بخشد.
همچنین پشتیبانی از ویژگی‌های LTE-Advanced و استقرار سلولهای کوچک  می‌تواند ظرفیت را در مناطق پرجمعیت و نقاط پرتردد افزایش دهد.
اما تمرکز باند متمرکز (BBU-Pool) به اندازه کافی پیش نمی‌رود.
برای دستیابی به پتانسیل کامل صرفه‌جویی در هزینه، مقیاس گزاری ظرفیت پویا، کیفیت بالاتر و ارائه سریع سرویسهای جدید، می‌بایست از یک معماری RAN مجازی (vRAN) استفاده کنند. در مدل vRAN،
BBU 
مجازی شده است.
vBBU
که همان واحدهای باند پایه‌ی مجازی هستند، در چندین سیستم عامل NFV در سخت افزار استاندارد x86 مستقر شده و در مراکز داده متمرکز تلفیق می‌شوند، در حالی که واحدهای رادیویی از راه دور (RRH) در سایتهای سلول در لبه باقی می‌مانند.
vRAN 
از سخت افزار استاندارد سرور استفاده می‌کند که به طور مقرون به صرفه پردازش، حافظه و منابع ورودی و خروجی را با تقاضای خود، درخواست می‌کند و ظرفیت RAN را با هوش مصنوعی تغییر داده تا کیفیت و قابلیت اطمینان خدمات را به طور قابل توجهی بهبود بخشد.
بسته به نحوه تقسیم عملکردهای eNodeB، معماری vRAN همچنین امکان انتقال اترنت و IP را فراهم می‌کند، که به ارائه‌دهندگان خدمات گزینه‌های مقرون به صرفه‌تری برای انتقال fronthaul می‌دهد \cite{vran}.
\begin{figure}%[H]
	\centering
	\includegraphics[width=0.8\textwidth]{./fig/vran}
	\caption{ساختار شبکه ی \lr{vRAN} \cite{vran}}
	\label{fig:vran}
\end{figure}


\subsection{مقدمه ای بر ORAN}
شبکه دسترسی رادیویی باز از ترکیب C-RAN و xRAN و در برخی جاها از ترکیب C-RAN و vRAN بدست آمده است.
معماری \lr{ORAN} برای ایجاد زیرساختهای \lr{RAN} نسل بعدی طراحی شده است.
معماری \lr{ORAN} با تکیه بر اصول هوشمندی و باز بودن، پایه و اساس ساخت \lr{RAN} مجازی بر روی سخت افزار آزاد، با کنترل رادیویی ایجاد شده توسط هوش مصنوعی است که توسط اپراتورهای سراسر جهان پیش بینی شده است.
این معماری بر روی رابطهای استاندارد و تعریف شده ای بنا شده است تا یک زنجیره اکوسیستم با قابلیت باز ایجاد کند که دارای پشتیبانی کامل از استانداردهای تبلیغ شده توسط \lr{3GPP} و سایر سازمانهای استاندارد صنعت فراهم شود.
\begin{figure}%[H]
  \centering
    \includegraphics[width=0.8\textwidth]{./fig/oran1}
  \caption{ساختار شبکه ی \lr{ORAN} \cite{oranWP}}
  \label{fig:ORAN}
\end{figure}
اتحاد \lr{ORAN} در جستجوی چشم انداز باز بودن و هوشمندی برای شبکه‌های بی سیم نسل بعدی و فراتر از آن است\cite{oranWP}.
این دو ویژگی مهم در ادامه مورد بررسی قرار گرفته شده است.
\begin{itemize}
\item \textbf{باز بودن}:
ایجاد یک \lr{RAN} مقرون به صرفه نیاز به باز بودن ارتباط‌ها دارد.
رابطهای باز برای فعال کردن فروشندگان و اپراتورهای کوچکتر به سرعت میتوانند خدمات خود را معرفی کنند و یا اپراتورها را قادر می‌سازد تا شبکه را متناسب با نیازهای منحصر به فرد خود تنظیم کنند.
رابطهای باز همچنین استقرار چند سازنده ای را قادر می‌سازد و اکوسیستم تأمین کننده رقابتی تر و پر جنب و جوش بیشتری را ایجاد می‌کند.
 همچنین نرم‌افزارهای منبع باز و طرحهای مرجع سخت افزار باعث نوآوری سریعتر و دموکراتیک تر می‌شود.
 \item \textbf{هوشمندی}
 شبکه‌ها با ظهور برنامه 5G پیچیده تر و متراکم تر شده و خواستار برنامه‌های غنی تر می‌شوند.
 برای کاستن این پیچیدگی نمیتوان از ابزارهای سنتی انسانی  برای استقرار، بهینه سازی و بهره برداری از شبکه استفاده کرد.
 در نتیجه، شبکه‌ها باید خود متحرک شوندتا بتوانند از فن آوریهای جدید مبتنی بر یادگیری برای خودکارسازی عملکرد شبکه‌های عملیاتی و کاهش \lr{OPEX} استفاده کنند.
 اتحاد \lr{ORAN} تلاش خواهد کرد تا از تکنیکهای یادگیری عمیق در حال ظهور استفاده کند تا بتواند هر لایه از معماری \lr{RAN}  را به طور هوشمند پیاده سازی کند.
 پیاده سای هوشمند هم در مولفه‌ها و هم در سطح شبکه اعمال می‌گردد و منجر به تخصیص دینامیکی منابع رادیویی و بهینه سازی بازدهی شبکه می‌گردد.
 همراه با رابطهای باز \lr{ORAN}، اتوماسیون حلقه بسته بهینه شده با هوش مصنوعی دست یافتنی است و دوره جدیدی را برای عملیات شبکه امکان پذیر می‌کند.
\end{itemize}
در ادامه ویژگی های این ساختار را بررسی می‌نماییم.
\begin{itemize}
\item \textbf{روشهای هوش مصنوعی \lr{AI}
 \LTRfootnote{Artificial Intelligent}
  منجر به هوشمندسازی بخش رادیویی با استفاده از نرم‌افزار تعریف شده \LTRfootnote{Software Defined} 
  می‌شود:}
   مفهوم 
  \lr{SDN}
  \LTRfootnote{software defined network}
  که مبنی بر جداسازی 
   بخش صفحه ی کنترل \lr{CP} از
   صفحه‌ی کاربر 
\lr{UP}
می‌باشد، در ساختار 
\lr{ORAN}
مورد بررسی قرار می‌گیرد.
این جداسازی منجر به بهبود 
\lr{RRM}
برای استفاده از زمان غیر واقعی و زمان نزدیک به واقعی در کنترلگر هوشمند شبکه ی دسترسی رادیویی \LTRfootnote{RAN Intelligent Controller} \lr{RIC} 
با استفاده از رابط‌های 
\lr{A1}
و
\lr{E2}
 می‌‌گردد.
همچنین 
منجر به جداسازی 
 \lr{CU}
 از 
 \lr{CP/UP}
 می‌شود
 که از طریق رابط \lr{E1} در \lr{3GPP } توسعه می‌یابد.
\item \textbf{مجازی سازی بخش \lr{RAN}}:
 ابری سازی RAN یکی از اصول مهم ساختار 
 \lr{ORAN}
  می‌باشد.
 اپراتورها برای پشتیبانی از شکافهای مختلف در شبکه، الزامات NFVI/VIM را برای تقویت سیستم عامل مجازی ارائه می‌دهند.
 به عنوان مثال: لایه ی بالا بین PDCP و RLC تقسیم می‌شود و لایه ی پایین در PHY تقسیم می‌شود.
\item \textbf{رابطهای باز}:
معماری مرجع ORAN بر روی مجموعه ای از رابطهای کلیدی بین چندین جزء جدا شده ی RAN ساخته شده است.
اینها شامل رابطهای \lr{3GPP} پیشرفته 
(
\lr{F1}،
\lr{W1}،
\lr{E1}،
\lr{X2}،
\lr{Xn}
)
 برای قابلیت همکاری بین چندین شرکت مختلف تولید کننده است.
رابطهای مشخص شده \lr{ORAN Alliance} شامل یک رابط fronthaul باز بین DU و RRU، رابط \lr{E2} و یک رابط \lr{A1} بین لایه 
Orchestration/NMS 
است که شامل عملکرد
 غیر واقعی زمانی \LTRfootnote{non real time RIC} و عملکرد eNB / gNB حاوی عملکرد RIC نزدیک به زمان واقعی 
\LTRfootnote{near-real time RIC} 
است.
\item \textbf{سخت افزار جعبه سفید}:
برای بهره مندی کامل از مقیاسی از اقتصاد ارائه شده توسط یک رویکرد  محاسباتی باز، \lr{O-RAN Alliance } 
طرحهای مرجع 
سخت افزاری و ایستگاه پایه به صورت جعبه سفید با کارایی بالا را مشخص می‌کند. 
سیستم عاملهای مرجع از یک رویکرد جدا شده پشتیبانی میکنند و نقشه‌های مفصلی را برای معماری سخت افزار و نرم‌افزار ارائه می‌دهند تا هم BBU و RRU را فعال کنند. 
\item \textbf{نرم‌افزار منبع باز}:
اتحادیه ORAN ارزش انجمنهایی که منابع باز ارا‌ئه می‌دهند را درک کرده   
 و از آنها پشتیبانی می‌کند.
 بسیاری از مؤلفه‌های معماری ORAN به صورت منبع باز از طریق جوامع موجود تحویل داده می‌شود.
 این مؤلفه‌ها عبارتند از: کنترلر هوشمند RAN، پشته پروتکل، پردازش لایه PHY و بستر مجازی سازی.
  چارچوب نرم‌افزار منبع باز ORAN نه تنها رابطهای 
(
\lr{F1}،
\lr{W1}،
\lr{E1}،
\lr{E2}،
\lr{X2}،
\lr{Xn}
)
  را پیاده سازی می‌کند، بلکه انتظار دارد که طراحی مرجع را برای نسل بعدی RRM با هوش جاسازی شده ارائه دهد تا RIC را امکان پذیر کند.
\end{itemize}

\lr{ORAN}،
 المانهای شبکه ی دسترسی رادیویی را مجازی می‌کند، آنها را جدا کرده و رابطهای باز مناسب را 
برای اتصال این عناصر
تعیین می‌کند. همچنین، 
\lr{ORAN}
از روشهای یادگیری ماشین برای هوشمندسازی لایه‌های 
\lr{RAN}
 استفاده می‌نماید. 
 در ساختار نوآورانه ی 
 \lr{ORAN}
 نرم‌افزار قابل برنامه ریزی 
 \lr{RAN}
 از سخت افزار جدا می‌شود.
  یکی از مهم ترین خصوصیات
  \lr{ORAN}
  رابط کاربری باز است که به اپراتورهای موبایل این قابلیت را می‌دهد تا بتوانند سرویسهای مورد نیاز خود را تعریف نمایند.

در ساختار
\lr{ORAN}،
واحد توزیع شده \lr{DU}،
نود منطقی می‌باشد که شامل لایه‌های 
\lr{RLC}
،
\lr{MAC}،
و
\lr{High-PHY}
است.
علاوه بر این، واحد مرکزی 
\lr{CU}
نود منطقی است که شامل لایه‌های 
\lr{RRC}،
\lr{SDAP} 
و 
\lr{PDCP}
می‌باشد.
نود منطقی واحد رادیویی
\lr{RU}
نیز، شامل لایه‌ی 
\lr{LOW-PHY}
و بخش پردازش رادیویی می‌باشد.
\lr{ORAN}
،
رابطهایی از جمله رابط 
\lr{fronthaul}
باز را شامل می‌شود که بخش \lr{DU} را به \lr{RU} متصل می‌نماید
(رابط 
\lr{E2}). 
همچنین
 رابط \lr{A1}
 بین لایه ی 
  \lr{orchestration/NMS}
  که شامل 
  تابع غیر واقعی زمان است و 
  \lr{eNB/qNB}
  که شامل تابع نزدیک به زمان است. 


  با افزایش ترافیک تلفن همراه، شبکه‌های تلفن همراه و تجهیزاتی که آنها را اجرا می‌کند باید نرم‌افزاری تر، مجازی، انعطاف پذیر، هوشمند و کارآمدتر شوند.
اتحادیه ی ORAN متعهد است در حال تکامل شبکه‌های دسترسی رادیویی باشد که باعث می‌شود آنها نسبت به نسلهای قبل بازتر و باهوش تر شوند.
تجزیه و تحلیل در زمان واقعی که توسط سیستمهای یادگیری ماشین تعبیه شده است و ماژولهای پایانی هوش مصنوعی را هدایت می‌کند، باعث تقویت هوش شبکه ‌‌شود.
عناصر شبکه مجازی با رابطهای باز و استاندارد، جنبه‌های اصلی طرحهای مرجع توسعه یافته توسط اتحادیه ی ORAN خواهد بود.
فن آوریهای موجود از عناصر شبکه منبع باز و جعبه سفید، نرم‌افزار و اجزای سخت افزاری مهم این طرحهای مرجع خواهد بود.

\subsubsection{ساختار ORAN}
O-RAN Alliance~\footnote{https://www.o-ran.org} اخیراً طراحی یک معماری جدید
 RAN را برای تحقق بخشیدن به چشم انداز تبدیل RAN به باز، هوشمند، مجازی سازی شده و کاملاً تعامل پذیر راه اندازی کرده است. توانمندسازی چنین ویژگی‌هایی برای توانمندسازی نسل بعدی شبکه‌های سلولی بی‌سیم برای پاسخگویی به نیازهای خدمات متنوع به روشی مقرون‌به‌صرفه حیاتی است. مفهوم ORAN مزایای مفاهیم C-RAN و vRAN را ترکیب و تکامل می دهد. ORAN با تکیه بر تلاش های قبلی برای ابری سازی و متمرکز سازی واحدهای باند پایه معرفی شده توسط C-RAN و جداسازی نرم افزار از سخت افزار فعال شده توسط vRAN، می آید تا با تعریف رابط های باز استاندارد بین، مشکلات قفل فروشنده و پیاده سازی اختصاصی را حل کند. اجزای RAN گشودگی ارائه شده توسط ORAN اجازه می دهد تا یک اکوسیستم زنجیره تامین با چند فروشنده را تقویت کند. علاوه بر باز بودن، ORAN هوش شبکه را از طریق ادغام AI/ML در اجزای RAN ارتقا می دهد.
 
 \begin{figure}%[htbp]
 	\centering
 	%\resizebox{\linewidth}{!}{
 	\resizebox{7.5cm}{!}{
 		\includegraphics{img/oran_last_new.pdf}
 	}
 	\caption{ساختار ORAN}
 	\label{fig:g11}
 \end{figure}

ORAN با تقسیم RAN خود به چندین مؤلفه کاربردی، یک RAN هوشمند و همه کاره ایجاد می کند.
برخلاف معماری C-RAN که دارای دو واحد RAN است، یعنی واحد رادیویی و باند پایه، O-RAN شامل سه واحد است: واحد رادیویی (O-RU)، واحد توزیع شده (O-DU) و واحد مرکزی (O-CU).
تصویر \ref{fig:g11} معماری ORAN را نشان می دهد.
معماری منطقی ORAN شامل سمت رادیویی، سمت مدیریتی و سمت ابری است.
\begin{itemize}
	\item بخش رادیویی:
	در سمت رادیویی از لایه‌های منطقی مختلف، از جمله 
	O-RU، O-DU، O-CU  
	و
	کنترل‌کننده هوشمند رادیویی نزدیک به زمان واقعی (near RT RIC) تشکیل شده است.
O-RU شامل لایه فرکانس رادیویی (RF) و لایه فیزیکی پایین (PHY) است، در حالی که O-DU عملکردهای لایه های PHY بالا، کنترل دسترسی متوسط (MAC) و کنترل پیوند رادیویی (RLC) را ارائه می دهد.

فرانتهال باز (Open-FH) رابط بین O-RU و O-DU است. رابط Open-FH شامل یک صفحه هماهنگ سازی کاربر کنترل (CUS-plane) و یک صفحه مدیریت (M-plane) است. رابط Open-FH M-plane O-RU را برای قابلیت های خطا، پیکربندی، حسابداری، عملکرد و امنیت (FCAPS) به مدیریت خدمات و هماهنگ سازی (SMO) متصل می کند.	

O-CU به دو گره منطقی صفحه کاربر (O-CU-UP) و صفحه کنترل (O-CU-CP) تقسیم می شود. O-CU-UP شامل پروتکل تطبیق داده های سرویس (SDAP) است که کیفیت خدمات حامل های رادیویی را مدیریت می کند. و بخش صفحه کاربر از پروتکل همگرایی داده های بسته (PDCP) که عملکردهای انتقال داده، تکرار بسته ها، رمزگذاری و حفاظت از یکپارچگی و غیره را فراهم می کند.

O-CU-CP میزبان لایه کنترل منابع رادیویی (RRC) است که چرخه عمر اتصال و صفحه کنترل پروتکل PDCP را کنترل می کند.

\item بخش مدیریتی:
سمت مدیریت شامل چارچوب SMO است. در SMO، RIC غیر RT نقش مهمی دارد \cite{ORANArch}.
RIC غیر RT رویدادها و مدیریت منابع را با زمان پردازش حداقل 1 ثانیه مدیریت می کند. مدیریت منابع شامل
بهینه سازی در RAN، استفاده از سیاست ها و مدل های ML برای RIC های نزدیک به RT برای افزایش عملکرد سیستم.
علاوه بر این، مدیریت چرخه حیات را برای اجزای شبکه فراهم می کند. علاوه بر این، پیکربندی و سایر جنبه های حیاتی یک شبکه را انجام می دهد.
\item بخش ابری:
این یک پلت فرم محاسبات ابری به نام O-Cloud را نشان می دهد که می تواند میزبان اجزای معماری O-RAN باشد که در شکل \ref{fig:g12} نشان داده شده است.
پلتفرم O-Cloud شامل زیرساخت سخت افزاری و فناوری های مجازی سازی است که برای فعال کردن نرم افزار O-RAN و جداسازی سخت افزار \cite{ORANSecOcloud} لازم است. زیرساخت سخت افزار مجموعه ای از سرورهای تجاری خارج از قفسه (COTS) است که منابع مدیریت محاسبات، ذخیره سازی، شبکه و سخت افزار را فراهم می کند.
توابع شبکه RAN را می توان به ترتیب به صورت توابع شبکه مجازی شده (VNF) یا توابع شبکه بومی ابری (CNF) در حال اجرا بر روی ماشین های مجازی (VM) یا کانتینرها مستقر کرد. لایه مجازی سازی پلت فرم O-Cloud شامل اجزای نرم افزاری پشتیبانی کننده (مانند سیستم عامل ها، هایپروایزرها و موتورهای کانتینری) برای اجرای VNF و CNF های RAN است.
O-Cloud همچنین از شتاب دهنده های سخت افزاری و یک لایه انتزاعی شتاب (AAL) پشتیبانی می کند که مجموعه ای از API های باز را برای بارگذاری عملکردهای شبکه O-RAN با شتاب سخت افزاری تعریف می کند.
SMO از طریق رابط $\text{O}2$ به O-Cloud برای ارائه منابع پلت فرم و قابلیت های مدیریت حجم کار \cite{ORANArch} متصل می شود.
\begin{figure}
	\centering
	%\resizebox{\linewidth}{!}{
	\resizebox{7.5cm}{!}{
		\includegraphics{img/ocloud_new.pdf}
	}
	\caption{ ساختار O-Cloud}
	\label{fig:g12}
\end{figure}

\end{itemize}

\subsubsection{آسیب پذیری ها و تهدیدها در معماری ORAN}
در این بخش هدف، صحبت در مورد آسیب پذیری‌ها و تهدیدها در معماری ORAN می‌باشد.

باز بودن و تفکیک معماری ORAN راه را برای یک وضعیت امنیتی تقویت شده برای شبکه های تلفن همراه آینده هموار می کند و انطباق با استانداردهای امنیتی را تسهیل می کند و چابکی امنیتی، سازگاری و انعطاف پذیری را تقویت می کند. با این حال، با این مزایا، پتانسیل افزایش سطح حمله ارائه شده توسط مؤلفه‌ها و رابط‌های جدید معماری ORAN \cite{ORANSec} به وجود می‌آید. در ادامه این بخش، آسیب‌پذیری‌ها و تهدیدات اصلی علیه سیستم ORAN را با در نظر گرفتن نه تنها مواردی که توسط فن‌آوری‌های جدید و اصول طراحی معماری ORAN به ارمغان می‌آورند، بلکه همچنین مسائل رایج امنیتی \lr{5G} RAN را مورد بحث قرار می‌دهیم.
این آسیب پذیری‌ها در موارد زیر خواهد بود:
\begin{itemize}
	\item \lr{Near-RT RIC} :
	از طریق رابط های استاندارد و پشتیبانی سخت افزاری، \lr{Near-RT RIC} یک پلت فرم ایمن و قابل اعتماد برای میزبانی xApps فراهم می کند. xApps مستقل از \lr{Near-RT RIC} هستند و ممکن است توسط یک فروشنده شخص ثالث عرضه شوند. \lr{Near-RT RIC} و xApps می توانند منبع تهدیدات امنیتی مختلف باشند \cite{ORANSec}.
	یک xApp مخرب یا در معرض خطر با دستکاری داده‌های جمع‌آوری‌شده از گره‌های E2 (O-DU, O-CU-CP و O-CU-UP) و رابط A1، این پتانسیل را دارد که بر ارائه خدمات برای یک مشترک، گروهی از مشترکین یا یک منطقه جغرافیایی خاص تأثیر منفی بگذارد.
	همچنین خطر دسترسی غیرمجاز به گره‌های E2 و Near-RT-RIC، سوء استفاده از عملکردهای RAN و ایجاد اثرات مضر برای سیستم کلی را معرفی می‌کند.
	نشت داده های حساس
	 (به عنوان مثال، شناسایی و مکان 
	 UE
	 ) 
	تهدید دیگری است که می تواند از برنامه های مخرب/در معرض خطر نشات بگیرد.
	افشای اطلاعات حساس نه تنها باعث نقض حریم خصوصی می شود، بلکه ممکن است منجر به حملات دیگری مانند جعل هویت و حملات ردیابی UE شود.
	
\item SMO :
از نظر امنیت، SMO بسیار مهم است زیرا یک آسیب‌پذیری موفقیت‌آمیز در SMO می‌تواند نقطه ورود برای حمله به اجزای O-RAN و انجام حرکت جانبی در شبکه باشد.
در واقع، رویه‌های احراز هویت و مجوز اجرا شده نادرست به مهاجم اجازه می‌دهد داده‌های ذخیره شده در SMO را افشا و تغییر دهد، به عملکردهای SMO و داده‌های آن‌ها دسترسی کامل داشته باشد، اجزای O-RAN را دستکاری کند و اطلاعات حساس O-RAN را بدزدد.
برای مثال، دسترسی غیرمجاز به عملکرد RIC غیر RT از طریق SMO ممکن است منجر به ردیابی UE یا صدور یک خط مشی نادرست برای RIC نزدیک به RT شود.
علاوه بر این، SMO و عملکردهای آن، به ویژه RIC غیر RT، می توانند قربانی حملات DoS بیش از حد شوند، که در دسترس بودن آنها را مختل کرده یا عملکرد آنها را کاهش دهند.
در واقع، یک حمله DoS علیه Non-RT-RIC مانع از توانایی آن در تجزیه و تحلیل و نظارت بر سیستم شبکه، به‌روزرسانی خط‌مشی‌های A1 و تنظیم قوانین کنترل در RIC نزدیک به RT~\cite{ORANSec} می‌شود.
rApps یکپارچه شده در Non-RT-RIC نگرانی های امنیتی مشابهی را ایجاد می کند که برای xApps مورد بحث قرار گرفت.
\item 
O-RU/O-DU و Open-FH :
O-RU ها می توانند هدف تهدید ایستگاه پایه کاذب (FBS) باشند، جایی که مهاجم به عنوان یک ایستگاه پایه قانونی ظاهر می شود تا حمله Man-in-The-Middle (MiTM) را بین UEs و شبکه تلفن همراه فعال کند.

سه سناریوی احتمالی حمله FBS در O-RU قابل تشخیص است~\cite{ORANSec}، یعنی: ربودن fronthaul ، استخدام یک O-RU مستقل، و دسترسی فیزیکی غیرمجاز به O-RU
.

در سناریوی هواپیماربایی، مهاجم یک سیستم FBS را به رابط Open-FH یک O-RU عملیاتی متصل می کند و با اتصال O-RU به رابط هوایی، یک حمله FBS را انجام می دهد.
در سناریوی مستقل O-RU، O-RU مورد حمله عملیاتی نیست اما برای یک مهاجم برای ادغام در یک سیستم FBS قابل دسترسی است.
در آخرین سناریو، سایر اجزای O-RU غیر از رابط Open-FH توسط مهاجم برای اتصال O-RU هدف به یک سیستم FBS قابل دسترسی است.
وجود FBS در شبکه چندین خطر را برای کاربر مشترک ایجاد می کند، از جمله سرقت اطلاعات کاربر، تغییر و تغییر مسیر داده های ارسال شده، به خطر انداختن حریم خصوصی کاربر و ردیابی کاربران. همچنین ممکن است به نفوذ O-DU و فراتر از آن در CN و راه اندازی حملات DoS برای از دست دادن سرویس یا کاهش عملکرد آن کمک کند.
با توجه به اینکه O-DU و O-RU می توانند توسط فروشندگان متمایز ارائه شوند، ممکن است مدل های امنیتی ناهمگون برای آنها اعمال شود که در نتیجه سطوح امنیتی متفاوتی ایجاد می شود. نقش کلیدی O-DU در ایجاد ترافیک مدیریت بین سیستم مدیریت و O-RU خطر دسترسی غیرمجاز به سیستم های شمال به خارج از O-DU مانند RIC ها از طریق رابط
 Open-FH
 ایجاد می کند ~\cite{ORANSec}
 .
علاوه بر این، یک رابط Open-FH محافظت نشده، حملات MiTM را بر روی M-plane یا CUS-plane تسهیل می کند. در نتیجه، مهاجم می تواند دستکاری و افشای داده ها و همچنین حملات DoS را انجام دهد. برای مثال، یک دستگاه غیرمجاز در رابط Open-FH اترنت L1 می‌تواند یک حمله سیل‌آمیز را راه‌اندازی کند، که باعث عدم دسترسی یا کاهش عملکرد عناصر شبکه قانونی در رابط Open-FH شود.
\item O-Cloud
:
پلتفرم O-Cloud در معماری ORAN خطرات امنیتی ابر مشترکی دارد که از جنبه های مختلف پشته ابری ناشی می شود.
ممکن است حملات نرم افزاری مختلفی مانند حملات نقص نرم افزار، دسترسی به یک حساب معتبر و عدم احراز هویت در رابط های O-Cloud وجود داشته باشد.
علاوه بر این، ماشین‌های مجازی و کانتینرهایی که مؤلفه‌های ORAN ابری را در O-Cloud اجرا می‌کنند، می‌توانند توسط یک عامل مخرب به روش‌های مختلف مورد سوء استفاده قرار گیرند.

یک پیکربندی نادرست که امتیازات غیر ضروری را به VM/کانتینر می دهد ممکن است منجر به افزایش امتیاز و فرار از انزوا شود. مهاجمان می توانند VM/ظروف میزبانی مشترک را با بدافزار آلوده کنند، VMs/Containerهای مخرب جدید را روی هاست مستقر کنند، به سرور ریشه دسترسی داشته باشند و در نهایت کل سیستم را نابود کنند. همچنین امکان دسترسی غیرمجاز و دستکاری داده های حساس وجود دارد.
علاوه بر این، استقرار VMs/containerهای آسیب‌پذیر ممکن است خطر DoS را در منابع مشترک ایجاد کند. علاوه بر مشکل در دسترس نبودن، یک حمله DoS شناسایی نشده ممکن است باعث آسیب اقتصادی شود اگر مهاجم موفق شود با استفاده از قابلیت مقیاس‌بندی خودکار، آن را به یک حمله اقتصادی انکار پایداری (EDoS) تغییر شکل دهد. حملات زنجیره تامین تهدید دیگری علیه تصاویر VM/container است، که در آن مهاجم می‌تواند کد مخربی را تزریق کند یا داده‌های داخل تصویر ناامن را تغییر دهد و همچنین کلیدهای خصوصی و رمزهای عبور موجود در تصویر را استخراج کند. در نهایت، یک رابط O2 محافظت نشده بین O-Cloud و SMO خطر حمله MiTM را افزایش می دهد، خدمات و درخواست های دستکاری و افشا را ارائه می دهد. برای مثال، یک مهاجم می‌تواند درخواست‌های مهاجرت را تغییر دهد تا VMها/کانتینرها را خارج از مرزهای قانونی قرار دهد.
\item ماشین لرنینگ:
استفاده از تکنیک های ML در ORAN نه تنها اطلاعات مورد نظر را برای توانمندسازی عملکردهای RAN مستقل فراهم می کند، بلکه مسائل امنیتی جدی را نیز معرفی می کند~\cite{mimran2022evaluating}. در واقع، مدل‌های ML مستعد چندین حمله خصمانه هستند که به دشمن اجازه می‌دهد مدل ML را به تصمیم‌گیری نادرست، یادگیری مدل‌های اشتباه یا افشای اطلاعات خصوصی ترغیب کند~\cite{aisecme}. فریب دادن یک مدل ML به تصمیم‌گیری اشتباه را می‌توان با تغییر مجموعه داده‌های مورد استفاده برای آموزش مدل آفلاین، تزریق داده‌های جعلی به یک مدل یادگیری آنلاین، یا ایجاد نمونه‌های ورودی که می‌توانند از مدل آموخته‌شده در زمان ارائه فرار کنند، حاصل شود. رویکردهای یادگیری مشارکتی، مانند FL، مستعد حملات مسمومیت مدل هستند، جایی که یک عامل مخرب می‌تواند مدل جهانی را با دستکاری پارامترهای مدل محلی آن به خطر بیاندازد. علاوه بر این، FL در برابر حملات استنتاجی آسیب‌پذیر است که مهاجم را قادر می‌سازد تا داده‌های آموزشی محلی خصوصی را با اعمال نفوذ پارامترهای مدل محلی استنتاج کند.
بر اساس قابلیت دسترسی، حملات به مدل‌های ML را می‌توان به حملات جعبه سفید، جعبه سیاه و جعبه خاکستری طبقه‌بندی کرد~\cite{aisecme}. در واقع، زمانی که مهاجم به ترتیب به داده های آموزشی و پارامترها و معماری مدل مورد نظر دسترسی کامل، جزئی یا بدون دسترسی داشته باشد، حمله خصمانه جعبه سفید، جعبه خاکستری یا جعبه سیاه در نظر گرفته می شود. حمله جعبه سفید به دلیل فرض یک مهاجم با دانش کامل کمتر واقع بینانه تلقی می شود، که دستیابی به آن در سناریوهای دنیای واقعی دشوار است.
\end{itemize}
\section{مجازی سازی توابع شبکه}
برای بهبود سرویس دهی در نسل پنجم مخابرات، جداسازی المانهای نرم‌افزاری و سخت افزاری شبکه صورت گرفته است و به عنوان 
مجازی سازی توابع شبکه (\lr{NFV}) \LTRfootnote{network function virtualization}
معرفی شده است.
  حال توابع شبکه ی مجازی
  \lr{VNF}
  \LTRfootnote{Virtual network function}،
  بلوکهای توابع سیستم هستند.
در نسل پنجم مخابرات 
  انتظار می‌رود که
   میزبان چندین سرویس
   با نیازهای مختلف به طور همزمان
    باشند.
    ایده اصلی NFV جداسازی تجهیزات شبکه فیزیکی از توابع اجرا شده بر روی آنها است. این بدان معنی است که یک عملکرد شبکه - مانند فایروال - میتواند به عنوان نمونه ای از نرم‌افزارهای ساده به فراهم آورندگان سرویس (SP) \LTRfootnote{Service Provider} ارسال شود.
    این امر امکان ادغام بسیاری از انواع تجهیزات شبکه بر روی سرورهای با حجم بالا، سوئیچها و انبارها را فراهم می‌کند، که میتوانند در مراکز داده، نودهای شبکه توزیع شده و در محل کاربر نهایی قرار بگیرند.
    به این ترتیب، یک سرویس خاص میتواند به مجموعه ای از توابع شبکه مجازی (VNFs) تجزیه شود، که میتواند در نرم‌افزارهایی که روی یک یا چند سرور فیزیکی استاندارد در صنعت قرار دارند، اجرا شود.
    سپس VNF ها ممکن است در مکانهای مختلف شبکه (به عنوان مثال، با هدف معرفی خدمات هدفمند به مشتریان در یک موقعیت جغرافیایی خاص) جابجا شده و خدمات رسانی کنند، بدون اینکه لزوماً به خرید و نصب سخت افزار جدید نیاز داشته باشند.
    NFV به 
    ها SP 
   با انعطاف پذیری بیشتری وعده می‌دهد تا بتواند بیشتر قابلیتها و خدمات شبکه خود را به کاربران و سایر خدمات باز کنند و امکان استقرار یا پشتیبانی از سرویسهای جدید شبکه را  به طور سریعتر و ارزانتر داشته باشند تا بتوانند  سرویس بهتری داشته باشند.
   برای دستیابی به این مزایا، NFV مسیر را برای کاهش اختلافات در نحوه ارائه خدمات شبکه در مقایسه با عملکرد فعلی ایجاد می‌کند. خلاصه این ویژگیها به شرح زیر است
   \cite{NFV}.
 \begin{itemize}
  \item \textbf{جدا سازی بخش نرم‌افزار از سخت افزار}:
از آنجا که عنصر شبکه، ترکیبی از سخت افزارها و نرم‌افزارهای یکپارچه نخواهد بود، تکامل هر دو مستقل از یکدیگر می‌باشد.
که این ویژگی منجر به جداسازی زمان بندی توسعه و نگهداری نرم‌افزار و سخت افزار می‌گردد.
\item \textbf{استقرار عملکرد شبکه انعطاف پذیر:}
جدا کردن نرم‌افزار از سخت افزار به تنظیم مجدد و به اشتراک گذاری منابع زیرساختی کمک می‌کند،
بنابراین، سخت افزار و نرم‌افزار، باهمدیگر
میتوانند در زمانهای مختلف عملکردهای مختلفی را انجام دهد که به اپراتورهای شبکه کمک می‌کند تا خدمات جدید شبکه را سریعتر در همان پلت فرم فیزیکی مستقر کنند.
بنابراین،
مؤلفه‌ها را میتوان در هر دستگاه با قابلیت NFV در شبکه قرار داد و اتصالات آنها به روشی انعطاف پذیر تنظیم کرد.
\item \textbf{مقیاس گذاری پویا}:
  جداشدن عملکرد شبکه به اجزای نرم‌افزاری انعطاف پذیری بیشتری را برای  عملکرد واقعی VNF به روشی پویاتر، 
   با توجه به ترافیک واقعی که اپراتور شبکه برای تأمین ظرفیت نیاز دارد،
  فراهم می‌کند.
\end{itemize}  
در ادامه ساختار این شبکه به طور دقیق بیان می‌گردد.
 VNF
ها برای به اشتراک گذاشتن منابع مختلف فیزیکی و مجازی زیرساختها میتوانند مستقر و مجدداً تنظیم شوند، تا مقیاس پذیری و کارآمدی سیستم را تضمین کنند که منجر می‌شود SPها به سرعت سرویسهای جدید را در سیستم وارد کنند.
 به طور کلی، سه مؤلفه اصلی در NFV وجود دارد:
 خدمات، NFVI و مدیریت NFV و orchestration
 \LTRfootnote{NFV-MANO}
 که در شکل \eqref{fig:NFV} دیده می‌شود. 
 \begin{figure}
  \centering
    \includegraphics[width=0.8\textwidth]{./fig/NFV}
  \caption{ساختار NFV \cite{NFVArch}}
  \label{fig:NFV}
\end{figure} 
این مؤلفه‌ها به شرح زیر بیان می‌گردد\cite{NFVArch}.
\begin{enumerate}
\item 
خدمات: یک سرویس مجموعه ای از VNF ها است که میتوانند در یک یا چند ماشین مجازی پیاده سازی شوند.
در بعضی مواقع، VNF ها میتوانند در ماشینهای مجازی نصب شده در سیستم عامل یا سخت افزار بطور مستقیم نصب شوند. آنها توسط سرپرستان بومی یا مانیتورهای ماشین مجازی اداره می‌شوند.
معمولاً توسط یک سیستم مدیریت عناصر \LTRfootnote{Element Management System} (EMS)،
 که مسئولیت ایجاد، تنظیمات، نظارت، عملکرد و امنیت آن است، اداره می‌شود.
 EMS 
 اطلاعات ضروری مورد نیاز سیستم پشتیبانی عملیات \LTRfootnote{Operations Support System}(OSS) را در یک محیط SP فراهم می‌کند.
 OSS
  سیستم مدیریت عمومی است، که  همراه با سیستم پشتیبانی از تجارت 
\LTRfootnote{Business Support System}  
  (BSS)
 ، به ارائه دهندگان کمک می‌کند تا چندین سرویس ارتباطی از راه دور را به کار ببندند و مدیریت کنند.
  (به عنوان مثال سفارش، صورتحساب، تمدید، عیب یابی مشکل و غیره).
مشخصات NFV بر ادغام با راه حلهای موجود OSS / BSS متمرکز است.
\item NFVI
:
زیرساختهای NFV تمام منابع سخت افزاری و نرم‌افزاری را که شامل محیط NFV است، پوشش می‌دهد.
NFVI شامل اتصال شبکه بین مکانها، به عنوان مثال، بین
مراکز داده و ابرهای ترکیبی عمومی یا خصوصی است.
منابع فیزیکی به طور معمول شامل محاسبات، ذخیره سازی و سخت افزار شبکه است که وظیفه ی آن پردازش، ذخیره سازی و اتصال VNFها از طریق لایه مجازی سازی است و دقیقاً بالای سخت افزار قرار دارد و منابع فیزیکی را چکیده می‌کند (که به صورت منطقی تقسیم شده و به VNFها اختصاص می‌یابد).
هیچ راه حل خاصی برای استقرار NFV وجود ندارد. در عوض معماری NFV میتواند از یک لایه مجازی سازی موجود مانند Hypervisor با ویژگیهای استاندارد که منابع سخت افزاری را به راحتی استخراج می‌کند و آنها را به VNFها اختصاص می‌دهد، استفاده کند.
وقتی این پشتیبانی در دسترس نباشد، اغلب، لایه مجازی سازی از طریق یک سیستم عامل حاصل می‌شود که نرم‌افزاری را در بالای سرور غیر مجازی یا با اجرای یک VNF به عنوان یک برنامه اضافه می‌کند.
\item NFV-MANO
:
NFV-MANO
 از این موارد تشکیل شده است:
 orchestrator
،
 مدیران VNFs و مدیران زیرساخت مجازی.
 چنین بلوکی عملکردهای مورد نیاز برای کارهای مدیریتی را که برای VNFها اعمال می‌شود، به عنوان مثال تهیه و پیکربندی را  ارائه می‌دهد. 
 NFV-MANO شامل orchestration و مدیریت چرخه منابع فیزیکی یا مجازی است که از مجازی سازی زیرساختها و مدیریت چرخه VNFها پشتیبانی می‌کند.
 همچنین شامل بانکهای اطلاعاتی است که برای ذخیره اطلاعات و مدلهای داده استفاده می‌شود که ویژگیهای چرخه عمر توابع، خدمات و منابع را تعریف می‌کند.
 NFV-MANO روی کلیه وظایف مدیریتی مجازی سازی ویژه لازم در چارچوب NFV تمرکز دارد.
 علاوه بر این، این چارچوب رابطهایی را تعیین می‌کند که میتوانند برای ارتباطات بین مؤلفه‌‌های مختلف
  NFV MANO، 
 و همچنین هماهنگی با سیستمهای سنتی مدیریت شبکه (یعنی OSS و BSS) مورد استفاده قرار گیرند تا امکان عملکرد هر دو VNF و کارکردهای اجرا شده بر روی تجهیزات فراهم شود.
 به طور خلاصه، اگر برش شبکه با استفاده از فایروال و DPI مستقر شده باشد، آنگاه NFV-MANO وظیفه دارد بگوید این VNFها در کجای شبکه فیزیکی قرار دارند. همچنین این VNFها توسط EMS و همان MANO کنترل می‌شوند.
\end{enumerate}
\section{زیرساخت تعریف شده توسط نرم افزار}
زیر‌ساخت تعریف شده توسط نرم‌افزار (SDI\LTRfootnote{Software Defined Infrastructure})
تعریفی از زیرساختهای محاسبات فنی است که کاملاً تحت کنترل نرم افزار بدون دخالت اپراتور یا انسان است.
این عمل مستقل از هرگونه وابستگی خاص سخت افزاری عمل می کند و از لحاظ برنامه قابل توسعه است.
در رویکرد SDI، الزامات زیرساختی یک برنامه به صورت  الزامات کاربردی و غیر عملکردی تعریف شده است به گونه‌ای که می توان به طور خودکار سخت افزار کافی و مناسب برای تحقق این نیازها تهیه کرد.
این زیرساخت شامل شبکه‌ی تعریف شده‌ی نرم‌افزار و شبکه‌ی رادیویی دسترسی تعریف شده‌ی نرم‌افزار می‌باشد که در ادامه توضیح می‌دهیم.
\subsection{شبکه تعریف شده نرم‌افزار (SDN)}
بنیاد شبکه باز 
\LTRfootnote{Open Networking Foundation}
(ONF) 
یک مجموعه ای است که به توسعه، استاندارد سازی و تجاری سازی SDN 
\LTRfootnote{Software Defined Network} 
پرداخت.
ONF
 به طور صریح و دقیق SDN را بدین صورت تعریف کرد:
 شبکه تعریف شده توسط نرم‌افزار (SDN) یک معماری شبکه است که کنترل شبکه از ارسال جدا می‌شود و به طور مستقیم قابل برنامه ریزی است.
 SDN توسط دو ویژگی تعریف می‌شود، یعنی جدا شدن صفحه ی کنترل و داده و قابلیت برنامه ریزی در صفحه کنترل.
 با این وجود، هیچ یک از این دو امضای SDN در معماری شبکه کاملاً جدید نیستند
 \cite{SDN1}.
 SDN 
 در اصل یک الگوی شبکه سازی متمرکز است که در آن هوش شبکه (یعنی عملکرد کنترل یا صفحه کنترل) به طور منطقی در یک یا مجموعه ای از موجودیتهای کنترل (یعنی کنترل کننده‌های SDN) متمرکز می‌شود در حالی که صفحه ی انتقال داده،  ساده و چکیده شده برای برنامه‌‌های کاربردی می‌باشد و سرویسهای شبکه درخواست خود را از طریق کنترل کننده‌های SDN
 بیان میکنند.
 در حالی که در مورد هسته اصلی شبکه موبایل LTE، EPC صحبت میکنیم، مفهوم SDN برای دستیابی به جدایی واضح بین صفحات کنترل و کاربر در اشخاص SGW و PGW استفاده می‌شود.
 با تقسیم دروازه به این روش (یعنی از SGW به SGW-C و
SGW-U 
و از PGW به PGW-C و (PGW-U 
  مقیاس بندی این مؤلفه‌‌ها به طور مستقل امکان پذیر است و طیف وسیعی از گزینه‌‌های استقرار را نیز ممکن می‌کند.
  
 پروتکل مورد استفاده بین صفحه ی کنترل و صفحه ی کاربر میتواند یا افزونه پروتکل موجود OpenFlow باشد، که توسط گروه کاری بی سیم و موبایل ONF (WMWG)با رابطهای جدید، یعنی Sxa و Sxb ساخته می‌شود، که توسط  \lr{3GPP CUPS} تعریف و مشخص می‌شوند\cite{SDN2}. 
 جداسازی صفحه ی کنترل از کاربر منجر به کنترل بیشتر شبکه بوسیله ی برنامه می‌گردد که منجر به بهبود تنظیمات و کارآمدی سیستم می‌گردد.
 SDN
  با ساختار برنامه ریزی شده ی قوانین ترافیک، جایگزین امیدوار کننده ای برای فرماندهی ترافیک ارائه می‌دهد.
  ساختار SDN در شکل \eqref{fig:SDN} آورده شده است.
  \begin{figure}
  \centering
    \includegraphics[width=0.8\textwidth]{./fig/SDN}
  \caption{ ساختار SDN \cite{SDN3}}
  \label{fig:SDN}
\end{figure} 
  در این ساختار ۳ لایه ی مختلف وجود دارد که در ادمه بیان میکنیم\cite{SDN3}.
  \begin{enumerate}
  \item لایه ی برنامه :
  این لایه مجموعه ای از برنامه ‌های متمرکز بر خدمات شبکه را پوشش می‌دهد و آنها عمدتا برنامه‌‌های نرم‌افزاری هستند که با لایه کنترل ارتباط برقرار میکنند.
  \item لایه ی کنترل:
  به عنوان هسته اصلی SDN، لایه کنترل از یک کنترلر متمرکز تشکیل شده است که منطقاً نمای شبکه جهانی و پویا را حفظ می‌کند،  که از لایه برنامه درخواست می‌کند و دستگاه‌های شبکه را از طریق پروتکلهای استاندارد مدیریت می‌کند.
  \item لایه ی داده:
  این لایه، زیرساختها شامل سوئیچها، روترها و لوازم شبکه می‌باشد. در زمینه SDN، این دستگاه‌ها قابل برنامه ریزی هستند و از رابطهای استاندارد پشتیبانی میکنند.
  \end{enumerate}
\subsection{شبکه دسترسی رادیویی تعریف شده نرم‌افزار (SDRAN)}
SDRAN \LTRfootnote{Software Defined Radio Access Network}
یا شبکه‌ی دسترسی رادیویی تعریف‌ شده‌ی نرم‌افزار، یک بازنگری اساسی در لایه دسترسی رادیویی است.
SDRAN
یک صفحه‌ی کنترل متمرکز نرم‌افزار تعریف شده است برای بخش شبکه دسترسی رادیویی که ایستگاه‌های پایه را در یک مکان جغرافیایی داخلی، به عنوان یک ایستگاه پایه‌ی بزرگ مجازی با المانهای کنترلی مرکزی و رادیویی می‌باشد.
در این حالت، مفهوم SDRAN به مفهوم vRAN بسیار نزدیک است.
SDRAN
 صفحه کنترل و صفحه داده را در
 RAN 
 جدا می‌کند و تصمیمات کنترل را به صفحه کنترل متمرکز می‌کند. در معماری های رایج SDRAN، یک کنترل کننده مرکزی اطلاعات کل شبکه را جمع می کند و در سطح کلی برای هر عنصر صفحه داده تصمیم گیری می‌کند. این روش از سربار شدن تصمیم گیری در عناصر صفحه داده جلوگیری می‌کند و فرصتی را برای مدیریت انعطاف پذیر و هماهنگ در کل RAN فراهم می‌کند.
در تعریف چنین معماری، ما چارچوبی ایجاد می‌کنیم که از طریق آن یک شبکه جغرافیایی محلی می‌تواند به طور موثر توازن بار و مدیریت تداخل را انجام دهد و همچنین نرخ عملیاتی و یا هر هدف دیگر را به بهینه‌ترین مقدار خود برساند.
ما معتقدیم که طراحی تعریف شده توسط نرم افزار در RAN یک گام اساسی برای پشتیبانی از برش شبکه، به اشتراک گذاری RAN، مدیریت طیف انعطاف پذیر و سایر ویژگی های اصلی در شبکه های 5G خواهد بود.
امید بر این است که طراحی تعریف شده توسط نرم افزار در RAN 
(SDRAN) 
یک گام اساسی برای پشتیبانی از برش شبکه، به اشتراک گذاری RAN، مدیریت طیف انعطاف پذیر و سایر ویژگی های اصلی در شبکه های 5G خواهد بود
\cite{gudipati2013softran,yu2017hsdran}. 

 \section{برش شبکه}
 پیش بینی می‌شود شبکه‌‌های \lr{5G} چندین سرویس را با نیازهای مختلف به طور همزمان پشتیبانی کند.
 برش شبکه
 \LTRfootnote{Network Slicing}
به عنوان راه حلی برای چنین تقاضا در نظر گرفته شده است.
یک برش شبکه، یک شبکه منطقی \lr{end-to-end} است که خدمات  با نیازهای خاص را ارائه می‌دهد.
 چندین برش شبکه
در یک زیرساخت یکسان
  اجرا و مدیریت می‌شوند و
به طور مستقل کار میکنند.
برش شبکه
 با هدف تقسیم منطقی مجموعه توابع و منابع شبکه در یک نهاد شبکه در نظر گرفته شده است که مطابق با خواسته‌‌های فنی یا تجاری خاص می‌باشد.
 با خرد کردن یک شبکه فیزیکی به چندین شبکه منطقی، برش شبکه میتواند از خدمات متناسب با تقاضا برای سناریوهای برنامه مشخص در همان زمان با استفاده از همان شبکه فیزیکی پشتیبانی کند.
با استفاده از برش شبکه، منابع شبکه میتوانند به صورت پویا و کارآمد به برشهای شبکه منطقی با توجه به خواسته‌‌های QoS مربوطه اختصاص داده شوند\cite{NS1}. 

\begin{figure}
  \centering
    \includegraphics[width=0.8\textwidth]{./fig/NS}
  \caption{سه ساختار برش شبکه \cite{NS2}}
  \label{fig:NS}
\end{figure} 
پیاده سازیهای مختلفی از برش شبکه وجود دارد که شامل برش هسته ی شبکه، برش \lr{RAN} و برش هر دو بخش می‌باشد\cite{NS2}.
\begin{itemize}
\item \textbf{برش هسته:}
هسته ی شبکه (CN) \LTRfootnote{core network}
به عنوان برشهای شبکه، مجازی سازی می‌شوند که با ویژگیهایی مانند ویژگیهای قابل برنامه ریزی و قابل اعتماد بودن که شامل مدیریت حرکت و تأیید اعتبار می‌باشد.
برشهای شبکه فقط در CN وجود دارد.
  بنابراین، نه RAN و نه تجهیزات کاربر (UE) برای CNهای برش داده شده نیاز به تنظیم ویژه ندارند.
  در برش هسته ی شبکه، برش تنها در بخش هسته ی شبکه است و تمام واسطها و فرآیندها، بدون تغییر باقی میمانند
  به جز مواردی که در ابتدا UEها به شبکه‌‌ها وصل می‌شوند، زیرا UEها باید به برش صحیح CNها اختصاص داده شوند.
\item \textbf{
برش شبکه ی دسترسی رادیویی:
}
برخلاف برش CN،
برشهای RAN روی سخت افزار رادیویی و استخر منابع باند پایه، به نام یک سطح بی سیم، اجرا می‌شوند که دارای کشش کمتری نسبت به زیرساخت مجازی بالغ شده در CNها هستند.
با چند BS منطقی، برشهای RAN پارامترهای مختلفی از رابطهای هوا (به عنوان مثال، طول نماد، فاصله زیر حامل، طول پیشوند چرخه و پارامترهای درخواست تکرار خودکار هیبریدی \LTRfootnote{HARQ}) را اعمال می‌کند.
علاوه بر این، پارامترهای دیگری مانند انتخاب سلول و آستانه انتقال، و همچنین سیاستهای انتقال هماهنگ را میتوان برای هر برش تعریف کرد تا یک تجربه بی سیم برجسته را به کاربران ارائه دهد.
\item \textbf{برش هسته و شبکه ی دسترسی رادیویی}:
در این سناریو، هر برش از RAN به یک برش از هسته متصل می‌شود، بنابراین اپراتورها میتوانند یک شبکه منطقی انتهای به مشتریان ارائه دهند.
روش انتخاب برش همان روش برش RAN است، بنابراین کاربران پس از دسترسی به سیستم، نیازی به انتخاب برش CN ندارند.
این مدل از برش مزایای هر دو مدل از برش را باهم دارد.
 در نتیجه این روش برش، قادر به برنامه ریزی ویژگیهای CN و همچنین دارای قابلیت تغییر رابطهای هوایی RAN
 می‌باشد.
 
\end{itemize}
\section{مسئله‌ی کوله‌پشتی و بسته‌بندی جعبه}
در اینجا به دو مسئله‌ی کوله‌پشتی و بسته‌بندی جعبه می پردازیم. این دو مسئله NP-Hard هستند.
مسئله‌ی NP-hard را نمی توان در زمان چند جمله‌ای حل کرد و نیاز به روش ابتکاری است. در ادامه به بررسی این دو مسئله می‌پردازیم.
\subsection{مسئله‌ی کوله‌پشتی}
یکی از مسائل NP-Hard، مسئله ی کوله پشتی \LTRfootnote{knapsack}
می باشد. در این مسئله می خواهیم تعدادی شی با وزنهای مختلف را در تعدادی جایگاه با ظرفیت مشخص قرار دهیم.
هدف در این مسئله قرارگیری بیشترین تعداد اشیاء در این جایگاه‌ها می باشد.
حل این مسئله با استفاده از روش های مختلف صورت می‌گیرد. این مسئله بدین صورت فرمول بندی می‌شود.
\begin{subequations}
	\begin{alignat}{4}
		\max\limits_{\boldsymbol{x} }   \quad &   \sum_{j=1}^{N} p_j x_{j}\\
		\text{\lr{subject to}} \quad & \textstyle \sum_{j=1}^{N} r_{i,j}x_{j} \leq b_i  \quad i =1,...,m, \\
		&\textstyle  x_j \in \{0,1\} \label{eqmain}
	\end{alignat}
	\label{constraints2}
\end{subequations}
که در اینجا، m شرط، شرطهای کوله‌پشتی می‌باشد. در نتیجه این مسئله، کوله‌پشتی mبعدی می‌باشد\cite{chu1998genetic}.
$p_j$
ارزش گزاری شی jام می‌باشد.
همچنین 
برای حل این مسائل از الگوریتمهای branch-and-bound و cutting-plane و الگوریتمهای ابتکاری استفاده می‌شود.
\subsection{مسئله‌ی بسته‌بندی جعبه}
در این مسئله هدف قرار دادن تعدادی شیء در تعدادی جعبه با ظرفیت مشخص می‌باشد.
در مسئله ی بسته بندی جعبه \LTRfootnote{bin packing}
هدف کمینه کردن تعداد جعبه‌های ورودی با فرض اینکه همه‌ی اشیا در آن جا شوند.
حل این مسئله با استفاده از روش های مختلف صورت می‌گیرد. این مسئله بدین صورت فرمول بندی می‌شود.
\begin{subequations}
	\begin{alignat}{4}
		\min\limits_{\boldsymbol{x} }   \quad &  K= \sum_{j=1}^{n}  y_{j}\\
		\text{\lr{subject to}} \quad & K\geq 1 \\
		& \sum_{j=1}^{n}x_{ij} =1  \quad i =1,...,m, \\
		&  \sum_{i=1}^{m} r_{i} x_{ij} \leq b_j  \quad j=1,..,n \\
		&  x_{ij} \in \{0,1\}  \quad y_j \in \{0,1\} \quad \forall i,j \label{eqmain}
	\end{alignat}
	\label{constraints2}
\end{subequations}
در اینجا، $y_j$ تعداد جعبه‌های مورد استفاده است که درصورتی که  
$\sum_{i=1}^{n}x_{ij} \geq 1$
باشد، $y_j$ مقدار 1 می‌گیرد 
$x_{ij}$
نشان می‌دهد که شی i در جعبه‌ی j جا دارد.
و در غیر این صورت 0 است \cite{berkey1987two}.

\section{دستاوردهای پروژه}
در اینجا، هدف در نظرگیری ساختار رادیویی دسترسی باز در نسل پنجم و تخصیص منابع آن می‌باشد.
مسئله‌ی برش شبکه در بخش رادیویی و قرارگیری توابع مجازی شبکه برروی مراکز داده باهم در شبکه‌ی دسترسی رادیویی باز مورد بررسی قرار گرفته است. بخش رادیویی به صورت کامل مدل سازی شده و تاخیر و نرخ و پارامترهای دیگر بدست می آید. در اینچا فرض براین است که کاربران بر اساس سرویس مورد نیاز، دسته بندی می شوند و هدف تخصیص منابع فیزیکی محاسباتی به این برشهای اختصاص یافته به سرویسها می باشد.
برای حل این مسئله، ابتدا مسئله را به دو مسئله‌ی کوچکتر مختلف شکسته که در بخش اول، تخصیص برش شبکه به کاربران سرویسها و تخصیص توان در ساختار رادیویی باز حل شده و پس از آن، برشهایی از شبکه که به سرویس اختصاص داده شده را به مراکز داده نگاشت می‌دهیم.
در این مسئله، تاخیر و نرخ هر کاربر در سرویس مورد بررسی قرار گرفته شده و چالش تخصیص منابع که شامل برش بخش رادیویی به هر سرویس است و جاگیری توابع شبکه حل می‌شود.
الگوریتم بهینه  با استفاده از MOSEK و CVX بدست می‌آید، و الگوریتم استفاده شده در مسئله، یک روش ابتکاری و از جنس الگوریتمهای حریص می باشد. این مسئله به صورت متمرکز حل شده است. 
سپس مسئله به صورت ساده‌تر به دو مسئله‌ی بسته‌بندی جعبه و کوله‌پشتی نوشته شده و برای حالت دینامیکی متغیر با زمان با روش یادگیری تقویتی حل می‌شود.
\section{ساختار پروژه}
در این فصل مروری بر مفاهیم مورد استفاده در پروژه کردیم
در فصل دوم مروری بر ادبیات پیشین و خلاصه‌ای از مدل سیستم مقالات موجود، بیان می‌گردد. در این فصل ابتدا صورت مسئله‌ی برش شبکه، شبکه‌های دسترسی رادیویی باز و قرارگیری توابع شبکه در مراکز داده بررسی کرده سپس در مورد حل مسئله به روش دینامیکی صحبت می‌کنیم.
در فصل سوم مدل سیستم در نظر گرفته بیان می‌شود و صورت مسئله به نمایش گذاشته می‌شود و روشهای حل آن بیان می‌گردد. همچنین  نتایج شبیه سازی قرار داده می‌شود.
در فصل چهارم در صورت مسئله را ساده‌سازی کرده و با روش دینامیکی حل می‌نماییم.
در فصل پنجم نیز نتیجه گیری و کارهای آتی مورد نظر و پیشنهادات بیان می‌شود.   
\section{نتیجه گیری}
در این فصل ابتدا مروری بر تاریخچه ی مخابرات و ۵ نسل مخابراتی شد. سپس ساختارهای مختف دسترسی رادیویی به طور خلاصه بیان شد و در نتیجه ی آن ساختار CRAN که ساختار ابری است تعریف شد. سپس ساختار xRAN
مورد توجه قرار گرفت و در نهایت ساختار ORAN 
که ترکیب و تکاملی از CRAN و xRAN می‌باشد مورد توجه قرار گرفت.

بعد از بیان ساختارهای رادیویی، ساختار هسته ی شبکه را در نسل پنجم بیان کردیم که شامل 
NFV و SDN 
می‌باشد که منجر به جداسازی صفحه ی کنترل از کاربر می‌شود و سیستم هوشمندتر همراه با قابلیت برنامه ریزی بیشتر می‌گردد.
در ادامه برش شبکه در بخش رادیویی و هسته و هردو مورد توجه قرار گرفته شد.
سپس در مورد دو مسئله‌ی NP-Hard صحبت نمودیم. در نهایت در مورد دستاوردهای پروژه و ساختار فصلهای آتی صحبت می‌نماییم. 